
% Default to the notebook output style

    


% Inherit from the specified cell style.




    
\documentclass{article}

    
    
    \usepackage{graphicx} % Used to insert images
    \usepackage{adjustbox} % Used to constrain images to a maximum size 
    \usepackage{color} % Allow colors to be defined
    \usepackage{enumerate} % Needed for markdown enumerations to work
    \usepackage{geometry} % Used to adjust the document margins
    \usepackage{amsmath} % Equations
    \usepackage{amssymb} % Equations
    \usepackage[mathletters]{ucs} % Extended unicode (utf-8) support
    \usepackage[utf8x]{inputenc} % Allow utf-8 characters in the tex document
    \usepackage{fancyvrb} % verbatim replacement that allows latex
    \usepackage{grffile} % extends the file name processing of package graphics 
                         % to support a larger range 
    % The hyperref package gives us a pdf with properly built
    % internal navigation ('pdf bookmarks' for the table of contents,
    % internal cross-reference links, web links for URLs, etc.)
    \usepackage{hyperref}
    \usepackage{longtable} % longtable support required by pandoc >1.10
    

    
    
    \definecolor{orange}{cmyk}{0,0.4,0.8,0.2}
    \definecolor{darkorange}{rgb}{.71,0.21,0.01}
    \definecolor{darkgreen}{rgb}{.12,.54,.11}
    \definecolor{myteal}{rgb}{.26, .44, .56}
    \definecolor{gray}{gray}{0.45}
    \definecolor{lightgray}{gray}{.95}
    \definecolor{mediumgray}{gray}{.8}
    \definecolor{inputbackground}{rgb}{.95, .95, .85}
    \definecolor{outputbackground}{rgb}{.95, .95, .95}
    \definecolor{traceback}{rgb}{1, .95, .95}
    % ansi colors
    \definecolor{red}{rgb}{.6,0,0}
    \definecolor{green}{rgb}{0,.65,0}
    \definecolor{brown}{rgb}{0.6,0.6,0}
    \definecolor{blue}{rgb}{0,.145,.698}
    \definecolor{purple}{rgb}{.698,.145,.698}
    \definecolor{cyan}{rgb}{0,.698,.698}
    \definecolor{lightgray}{gray}{0.5}
    
    % bright ansi colors
    \definecolor{darkgray}{gray}{0.25}
    \definecolor{lightred}{rgb}{1.0,0.39,0.28}
    \definecolor{lightgreen}{rgb}{0.48,0.99,0.0}
    \definecolor{lightblue}{rgb}{0.53,0.81,0.92}
    \definecolor{lightpurple}{rgb}{0.87,0.63,0.87}
    \definecolor{lightcyan}{rgb}{0.5,1.0,0.83}
    
    % commands and environments needed by pandoc snippets
    % extracted from the output of `pandoc -s`
    \DefineVerbatimEnvironment{Highlighting}{Verbatim}{commandchars=\\\{\}}
    % Add ',fontsize=\small' for more characters per line
    \newenvironment{Shaded}{}{}
    \newcommand{\KeywordTok}[1]{\textcolor[rgb]{0.00,0.44,0.13}{\textbf{{#1}}}}
    \newcommand{\DataTypeTok}[1]{\textcolor[rgb]{0.56,0.13,0.00}{{#1}}}
    \newcommand{\DecValTok}[1]{\textcolor[rgb]{0.25,0.63,0.44}{{#1}}}
    \newcommand{\BaseNTok}[1]{\textcolor[rgb]{0.25,0.63,0.44}{{#1}}}
    \newcommand{\FloatTok}[1]{\textcolor[rgb]{0.25,0.63,0.44}{{#1}}}
    \newcommand{\CharTok}[1]{\textcolor[rgb]{0.25,0.44,0.63}{{#1}}}
    \newcommand{\StringTok}[1]{\textcolor[rgb]{0.25,0.44,0.63}{{#1}}}
    \newcommand{\CommentTok}[1]{\textcolor[rgb]{0.38,0.63,0.69}{\textit{{#1}}}}
    \newcommand{\OtherTok}[1]{\textcolor[rgb]{0.00,0.44,0.13}{{#1}}}
    \newcommand{\AlertTok}[1]{\textcolor[rgb]{1.00,0.00,0.00}{\textbf{{#1}}}}
    \newcommand{\FunctionTok}[1]{\textcolor[rgb]{0.02,0.16,0.49}{{#1}}}
    \newcommand{\RegionMarkerTok}[1]{{#1}}
    \newcommand{\ErrorTok}[1]{\textcolor[rgb]{1.00,0.00,0.00}{\textbf{{#1}}}}
    \newcommand{\NormalTok}[1]{{#1}}
    
    % Define a nice break command that doesn't care if a line doesn't already
    % exist.
    \def\br{\hspace*{\fill} \\* }
    % Math Jax compatability definitions
    \def\gt{>}
    \def\lt{<}
    % Document parameters
    \title{Landsat\_Preprocessing}
    
    
    

    % Pygments definitions
    
\makeatletter
\def\PY@reset{\let\PY@it=\relax \let\PY@bf=\relax%
    \let\PY@ul=\relax \let\PY@tc=\relax%
    \let\PY@bc=\relax \let\PY@ff=\relax}
\def\PY@tok#1{\csname PY@tok@#1\endcsname}
\def\PY@toks#1+{\ifx\relax#1\empty\else%
    \PY@tok{#1}\expandafter\PY@toks\fi}
\def\PY@do#1{\PY@bc{\PY@tc{\PY@ul{%
    \PY@it{\PY@bf{\PY@ff{#1}}}}}}}
\def\PY#1#2{\PY@reset\PY@toks#1+\relax+\PY@do{#2}}

\expandafter\def\csname PY@tok@gd\endcsname{\def\PY@tc##1{\textcolor[rgb]{0.63,0.00,0.00}{##1}}}
\expandafter\def\csname PY@tok@gu\endcsname{\let\PY@bf=\textbf\def\PY@tc##1{\textcolor[rgb]{0.50,0.00,0.50}{##1}}}
\expandafter\def\csname PY@tok@gt\endcsname{\def\PY@tc##1{\textcolor[rgb]{0.00,0.27,0.87}{##1}}}
\expandafter\def\csname PY@tok@gs\endcsname{\let\PY@bf=\textbf}
\expandafter\def\csname PY@tok@gr\endcsname{\def\PY@tc##1{\textcolor[rgb]{1.00,0.00,0.00}{##1}}}
\expandafter\def\csname PY@tok@cm\endcsname{\let\PY@it=\textit\def\PY@tc##1{\textcolor[rgb]{0.25,0.50,0.50}{##1}}}
\expandafter\def\csname PY@tok@vg\endcsname{\def\PY@tc##1{\textcolor[rgb]{0.10,0.09,0.49}{##1}}}
\expandafter\def\csname PY@tok@m\endcsname{\def\PY@tc##1{\textcolor[rgb]{0.40,0.40,0.40}{##1}}}
\expandafter\def\csname PY@tok@mh\endcsname{\def\PY@tc##1{\textcolor[rgb]{0.40,0.40,0.40}{##1}}}
\expandafter\def\csname PY@tok@go\endcsname{\def\PY@tc##1{\textcolor[rgb]{0.53,0.53,0.53}{##1}}}
\expandafter\def\csname PY@tok@ge\endcsname{\let\PY@it=\textit}
\expandafter\def\csname PY@tok@vc\endcsname{\def\PY@tc##1{\textcolor[rgb]{0.10,0.09,0.49}{##1}}}
\expandafter\def\csname PY@tok@il\endcsname{\def\PY@tc##1{\textcolor[rgb]{0.40,0.40,0.40}{##1}}}
\expandafter\def\csname PY@tok@cs\endcsname{\let\PY@it=\textit\def\PY@tc##1{\textcolor[rgb]{0.25,0.50,0.50}{##1}}}
\expandafter\def\csname PY@tok@cp\endcsname{\def\PY@tc##1{\textcolor[rgb]{0.74,0.48,0.00}{##1}}}
\expandafter\def\csname PY@tok@gi\endcsname{\def\PY@tc##1{\textcolor[rgb]{0.00,0.63,0.00}{##1}}}
\expandafter\def\csname PY@tok@gh\endcsname{\let\PY@bf=\textbf\def\PY@tc##1{\textcolor[rgb]{0.00,0.00,0.50}{##1}}}
\expandafter\def\csname PY@tok@ni\endcsname{\let\PY@bf=\textbf\def\PY@tc##1{\textcolor[rgb]{0.60,0.60,0.60}{##1}}}
\expandafter\def\csname PY@tok@nl\endcsname{\def\PY@tc##1{\textcolor[rgb]{0.63,0.63,0.00}{##1}}}
\expandafter\def\csname PY@tok@nn\endcsname{\let\PY@bf=\textbf\def\PY@tc##1{\textcolor[rgb]{0.00,0.00,1.00}{##1}}}
\expandafter\def\csname PY@tok@no\endcsname{\def\PY@tc##1{\textcolor[rgb]{0.53,0.00,0.00}{##1}}}
\expandafter\def\csname PY@tok@na\endcsname{\def\PY@tc##1{\textcolor[rgb]{0.49,0.56,0.16}{##1}}}
\expandafter\def\csname PY@tok@nb\endcsname{\def\PY@tc##1{\textcolor[rgb]{0.00,0.50,0.00}{##1}}}
\expandafter\def\csname PY@tok@nc\endcsname{\let\PY@bf=\textbf\def\PY@tc##1{\textcolor[rgb]{0.00,0.00,1.00}{##1}}}
\expandafter\def\csname PY@tok@nd\endcsname{\def\PY@tc##1{\textcolor[rgb]{0.67,0.13,1.00}{##1}}}
\expandafter\def\csname PY@tok@ne\endcsname{\let\PY@bf=\textbf\def\PY@tc##1{\textcolor[rgb]{0.82,0.25,0.23}{##1}}}
\expandafter\def\csname PY@tok@nf\endcsname{\def\PY@tc##1{\textcolor[rgb]{0.00,0.00,1.00}{##1}}}
\expandafter\def\csname PY@tok@si\endcsname{\let\PY@bf=\textbf\def\PY@tc##1{\textcolor[rgb]{0.73,0.40,0.53}{##1}}}
\expandafter\def\csname PY@tok@s2\endcsname{\def\PY@tc##1{\textcolor[rgb]{0.73,0.13,0.13}{##1}}}
\expandafter\def\csname PY@tok@vi\endcsname{\def\PY@tc##1{\textcolor[rgb]{0.10,0.09,0.49}{##1}}}
\expandafter\def\csname PY@tok@nt\endcsname{\let\PY@bf=\textbf\def\PY@tc##1{\textcolor[rgb]{0.00,0.50,0.00}{##1}}}
\expandafter\def\csname PY@tok@nv\endcsname{\def\PY@tc##1{\textcolor[rgb]{0.10,0.09,0.49}{##1}}}
\expandafter\def\csname PY@tok@s1\endcsname{\def\PY@tc##1{\textcolor[rgb]{0.73,0.13,0.13}{##1}}}
\expandafter\def\csname PY@tok@sh\endcsname{\def\PY@tc##1{\textcolor[rgb]{0.73,0.13,0.13}{##1}}}
\expandafter\def\csname PY@tok@sc\endcsname{\def\PY@tc##1{\textcolor[rgb]{0.73,0.13,0.13}{##1}}}
\expandafter\def\csname PY@tok@sx\endcsname{\def\PY@tc##1{\textcolor[rgb]{0.00,0.50,0.00}{##1}}}
\expandafter\def\csname PY@tok@bp\endcsname{\def\PY@tc##1{\textcolor[rgb]{0.00,0.50,0.00}{##1}}}
\expandafter\def\csname PY@tok@c1\endcsname{\let\PY@it=\textit\def\PY@tc##1{\textcolor[rgb]{0.25,0.50,0.50}{##1}}}
\expandafter\def\csname PY@tok@kc\endcsname{\let\PY@bf=\textbf\def\PY@tc##1{\textcolor[rgb]{0.00,0.50,0.00}{##1}}}
\expandafter\def\csname PY@tok@c\endcsname{\let\PY@it=\textit\def\PY@tc##1{\textcolor[rgb]{0.25,0.50,0.50}{##1}}}
\expandafter\def\csname PY@tok@mf\endcsname{\def\PY@tc##1{\textcolor[rgb]{0.40,0.40,0.40}{##1}}}
\expandafter\def\csname PY@tok@err\endcsname{\def\PY@bc##1{\setlength{\fboxsep}{0pt}\fcolorbox[rgb]{1.00,0.00,0.00}{1,1,1}{\strut ##1}}}
\expandafter\def\csname PY@tok@kd\endcsname{\let\PY@bf=\textbf\def\PY@tc##1{\textcolor[rgb]{0.00,0.50,0.00}{##1}}}
\expandafter\def\csname PY@tok@ss\endcsname{\def\PY@tc##1{\textcolor[rgb]{0.10,0.09,0.49}{##1}}}
\expandafter\def\csname PY@tok@sr\endcsname{\def\PY@tc##1{\textcolor[rgb]{0.73,0.40,0.53}{##1}}}
\expandafter\def\csname PY@tok@mo\endcsname{\def\PY@tc##1{\textcolor[rgb]{0.40,0.40,0.40}{##1}}}
\expandafter\def\csname PY@tok@kn\endcsname{\let\PY@bf=\textbf\def\PY@tc##1{\textcolor[rgb]{0.00,0.50,0.00}{##1}}}
\expandafter\def\csname PY@tok@mi\endcsname{\def\PY@tc##1{\textcolor[rgb]{0.40,0.40,0.40}{##1}}}
\expandafter\def\csname PY@tok@gp\endcsname{\let\PY@bf=\textbf\def\PY@tc##1{\textcolor[rgb]{0.00,0.00,0.50}{##1}}}
\expandafter\def\csname PY@tok@o\endcsname{\def\PY@tc##1{\textcolor[rgb]{0.40,0.40,0.40}{##1}}}
\expandafter\def\csname PY@tok@kr\endcsname{\let\PY@bf=\textbf\def\PY@tc##1{\textcolor[rgb]{0.00,0.50,0.00}{##1}}}
\expandafter\def\csname PY@tok@s\endcsname{\def\PY@tc##1{\textcolor[rgb]{0.73,0.13,0.13}{##1}}}
\expandafter\def\csname PY@tok@kp\endcsname{\def\PY@tc##1{\textcolor[rgb]{0.00,0.50,0.00}{##1}}}
\expandafter\def\csname PY@tok@w\endcsname{\def\PY@tc##1{\textcolor[rgb]{0.73,0.73,0.73}{##1}}}
\expandafter\def\csname PY@tok@kt\endcsname{\def\PY@tc##1{\textcolor[rgb]{0.69,0.00,0.25}{##1}}}
\expandafter\def\csname PY@tok@ow\endcsname{\let\PY@bf=\textbf\def\PY@tc##1{\textcolor[rgb]{0.67,0.13,1.00}{##1}}}
\expandafter\def\csname PY@tok@sb\endcsname{\def\PY@tc##1{\textcolor[rgb]{0.73,0.13,0.13}{##1}}}
\expandafter\def\csname PY@tok@k\endcsname{\let\PY@bf=\textbf\def\PY@tc##1{\textcolor[rgb]{0.00,0.50,0.00}{##1}}}
\expandafter\def\csname PY@tok@se\endcsname{\let\PY@bf=\textbf\def\PY@tc##1{\textcolor[rgb]{0.73,0.40,0.13}{##1}}}
\expandafter\def\csname PY@tok@sd\endcsname{\let\PY@it=\textit\def\PY@tc##1{\textcolor[rgb]{0.73,0.13,0.13}{##1}}}

\def\PYZbs{\char`\\}
\def\PYZus{\char`\_}
\def\PYZob{\char`\{}
\def\PYZcb{\char`\}}
\def\PYZca{\char`\^}
\def\PYZam{\char`\&}
\def\PYZlt{\char`\<}
\def\PYZgt{\char`\>}
\def\PYZsh{\char`\#}
\def\PYZpc{\char`\%}
\def\PYZdl{\char`\$}
\def\PYZhy{\char`\-}
\def\PYZsq{\char`\'}
\def\PYZdq{\char`\"}
\def\PYZti{\char`\~}
% for compatibility with earlier versions
\def\PYZat{@}
\def\PYZlb{[}
\def\PYZrb{]}
\makeatother


    % Exact colors from NB
    \definecolor{incolor}{rgb}{0.0, 0.0, 0.5}
    \definecolor{outcolor}{rgb}{0.545, 0.0, 0.0}



    
    % Prevent overflowing lines due to hard-to-break entities
    \sloppy 
    % Setup hyperref package
    \hypersetup{
      breaklinks=true,  % so long urls are correctly broken across lines
      colorlinks=true,
      urlcolor=blue,
      linkcolor=darkorange,
      citecolor=darkgreen,
      }
    % Slightly bigger margins than the latex defaults
    
    \geometry{verbose,tmargin=1in,bmargin=1in,lmargin=1in,rmargin=1in}
    
    

    \begin{document}
    
    
    \maketitle
    
    

    
    Landsat Data Preprocessing

Chris Holden

Friday, May 9th, 2014

\subparagraph{Overview:}

The Landsat family of satellites have been in space since 1972 and can
provide a rich history of the land surface of the Earth over the last
\textasciitilde{}40 years. Recent advances in computing hardware, free
access to the data, and new innovative cloud screening and atmospheric
correction algorithms have enabled researchers to ``data mine'' this
archive. Analyses that were previously only possible at a very coarse
resolution (e.g., AVHRR and MODIS) are beginning to be possible at the
30m Landsat resolution.

This workshop intends to teach good practices for acquiring, organizing,
preprocessing, and documenting your dense Landsat time series before
ingestion into the ``data mining'' algorithm of your choice.

Topics:

\begin{enumerate}
\def\labelenumi{\arabic{enumi}.}
\itemsep1pt\parskip0pt\parsep0pt
\item
  Search for and filter Landsat acquisitions using EarthExplorer
\item
  Submit order for ``Landsat Climate Data Record (CDR)'' product using
  ESPA
\item
  Download completed order onto GEO cluster
\item
  Organize and extract imagery from archives
\item
  Combine dataset bands into ``stacked'' images of common geographic
  extent
\item
  Run Fmask using custom parameters on images
\item
  Create spatial subset using extent or polygon ROI
\item
  Create ``preview'' images to help filter through preprocessed imagery
\item
  Summarize and document your dataset
\end{enumerate}

Notes on workshop

There are many ways to acquire and preprocess Landsat datasets and I
will be demonstrating but one of the many approaches. Where applicable,
I will demonstrate how to perform a step using a general method, and
then by using one of my tools developed to help automate the general
method for a specific workflow. The goal of showing both approaches will
be to give you access to my specialized scripts or programs while
helping you understand what they are automating.

    \subsection{Step 1. Search for and filter Landsat acquisitions using
EarthExplorer}

The USGS has many portals for filtering through Landsat data (e.g.,
GLOVIS, EarthExplorer, COVE tool, etc.), but the EarthExplorer tool is
one of the better ones. If you don't have an account with EarthExplorer,
you will need to create one before ordering data.

You will also need to know the WRS-2 path and row for the dataset you
wish to acquire. Further information on the WRS-2 reference system,
including a shapefile of all path and rows (download the descending
dataset for Landsat) is available here:

https://landsat.usgs.gov/worldwide\_reference\_system\_WRS.php

\subsubsection{EarthExplorer}

Navigate to:

http://earthexplorer.usgs.gov/

On the left side of the user interface you will see four tabs
representing components of the search workflow. Begin by skipping the
``Search Criteria'' tab and continuing to the ``Data Sets'' tab.

\subsubsection{Landsat CDR Datasets}

We will be submitting an order for the Landsat Climate Data Record (CDR)
product (http://landsat.usgs.gov/CDR\_ECV.php) so that the USGS can
perform the atmospheric correction and cloud masking preprocessing steps
for us before we download our data.

In the list of available datasets below, find ``Landsat CDR'', expand
the tab, and select both Landsat 7 ETM+ and Landsat 4-5 TM datasets.

\subsubsection{Filter}

Proceed by clicking the ``Additional Cirteria
\textgreater{}\textgreater{}'' button:

The next window of this tabbed interface will allow you to customize
your search result by various criteria for the Landsat 7 ETM+ and the
Landsat 4-5 TM datasets.

\begin{itemize}
\itemsep1pt\parskip0pt\parsep0pt
\item
  Input your WRS-2 path and row into the ``WRS Path'' and ``WRS Row''
  forms
\item
  Select ``Less than 80\%'' under the ``Cloud Cover'' option (this cloud
  cover is estimated from ACCA, not Fmask)
\item
  Ensure that the ``Station Identifier'' is set to ``All''
\item
  Select ``Processing Required'' and the ``L1T'' data type option for
  both sensors (e.g., ETM+ L1T and TM L1T). We select the ``L1T''
  product type because it is the only orthorectified product from the
  Landsat Level 1 Product Generation System (LPGS). Orthorectification
  is important to assure consistent geolocation of pixels on the land
  surface. We include images that have the ``Processing Required''
  status to include images that can be orthorectified to a L1T image,
  but have not yet been processed (and thus don't have the L1T metadata
  status in their database). For more information on LPGS, visit this
  URL: http://landsat.usgs.gov/Landsat\_Processing\_Details.php
\end{itemize}

Repeat this last step of adding filters for each Landsat sensor dataset
-- Landsat 7 ETM+ and Landsat 4-5 TM.

\subsubsection{Results}

Finally, click the ``Results'' button to submit your query.

\subsubsection{Export}

The last tab -- ``Results'' -- shows the Landsat imagery that met your
search query.

Click the button that says ``Click here to export your results'' to
export your results. Select ``Non-Limited Results'' under ``Export
Type''. I also recommend selecting the option ``Comma (,) Delimited''
for the ``Format''.

Remember to export your results from Landsat 4-5 TM and Landsat 7 ETM+.
The results will be retrievable via a URL sent to the email account you
registered with. In my case, my results are available at the following
URLs:

http://earthexplorer.usgs.gov/export/retrieve/?exportId=23721\&exportKey=1f457c3397af85c4c8b755355eb4eb52

http://earthexplorer.usgs.gov/export/retrieve/?exportId=23722\&exportKey=cc5e80832dca85903a510f4ac4ddb311

Manually click the links USGS provided you and download the ZIP files
from their website. Unzip these files to find two comma separated text
files.

    \subsection{Step 2: Submit order for ``Landsat Climate Data Record
(CDR)'' product using ESPA}

The two zip files containing our search results from the previous step
contain the metadata required to further filter through the Landsat
acquisitions. This includes information on the sun and sensor
geometries, the goodness of fit for the terrain correction, the date and
time of the acquisition, and more:

    \begin{Verbatim}[commandchars=\\\{\}]
{\color{incolor}In [{\color{incolor}52}]:} \PY{o}{\PYZpc{}\PYZpc{}}\PY{k}{bash}
         \PY{n}{tm}\PY{o}{=}\PY{o}{/}\PY{n}{projectnb}\PY{o}{/}\PY{n}{landsat}\PY{o}{/}\PY{n}{users}\PY{o}{/}\PY{n}{ceholden}\PY{o}{/}\PY{l+m+mi}{2014}\PY{n}{\PYZus{}Landsat\PYZus{}Preprocess}\PY{o}{/}\PY{l+m+mi}{1}\PY{n}{\PYZus{}EarthExplorer}\PY{o}{/}\PY{n}{LSR\PYZus{}LANDSAT\PYZus{}TM\PYZus{}23721}\PY{o}{.}\PY{n}{txt}
         
         \PY{n}{echo} \PY{o}{\PYZhy{}}\PY{n}{e} \PY{l+s}{\PYZdq{}}\PY{l+s}{Landsat Metadata Fields:}\PY{l+s+se}{\PYZbs{}n}\PY{l+s}{\PYZdq{}}
         
         \PY{n}{head} \PY{o}{\PYZhy{}}\PY{l+m+mi}{1} \PY{err}{\PYZdl{}}\PY{n}{tm} \PY{o}{|} \PY{n}{awk} \PY{o}{\PYZhy{}}\PY{n}{F} \PY{l+s}{\PYZsq{}}\PY{l+s}{,}\PY{l+s}{\PYZsq{}} \PY{l+s}{\PYZsq{}}\PY{l+s}{\PYZob{} max=30; }\PY{l+s+se}{\PYZbs{}}
         \PY{l+s}{    for (i=1; i\PYZlt{}=NF; i = i + 3) \PYZob{} }\PY{l+s+se}{\PYZbs{}}
         \PY{l+s}{        l1=(max \PYZhy{} length(\PYZdl{}i)); l2=(max \PYZhy{} length(\PYZdl{}(i + 1))); }\PY{l+s+se}{\PYZbs{}}
         \PY{l+s}{        print \PYZdl{}i sprintf(}\PY{l+s}{\PYZdq{}}\PY{l+s+si}{\PYZpc{}*s}\PY{l+s}{\PYZdq{}}\PY{l+s}{, l1, }\PY{l+s}{\PYZdq{}}\PY{l+s}{ }\PY{l+s}{\PYZdq{}}\PY{l+s}{) \PYZdl{}(i + 1) sprintf(}\PY{l+s}{\PYZdq{}}\PY{l+s+si}{\PYZpc{}*s}\PY{l+s}{\PYZdq{}}\PY{l+s}{, l2, }\PY{l+s}{\PYZdq{}}\PY{l+s}{ }\PY{l+s}{\PYZdq{}}\PY{l+s}{) \PYZdl{}(i + 2)\PYZcb{} }\PY{l+s+se}{\PYZbs{}}
         \PY{l+s}{    \PYZcb{}}\PY{l+s}{\PYZsq{}}
\end{Verbatim}

    \begin{Verbatim}[commandchars=\\\{\}]
Landsat Metadata Fields:

Result Number                 EE\_DISPLAY\_ID                 EE\_ORDERING\_ID
DOWNLOADABLE                  ON\_DEMAND                     Landsat Scene Identifier
Spacecraft Identifier         Sensor Mode                   Station Identifier
Day/Night                     WRS Path                      WRS Row
WRS Type                      Date Acquired                 Start Time
Stop Time                     Sensor Anomalies              Acquisition Quality
Quality Band 1                Quality Band 2                Quality Band 3
Quality Band 4                Quality Band 5                Quality Band 6
Quality Band 7                Cloud Cover                   Cloud Cover Quad Upper Left
Cloud Cover Quad Upper Right  Cloud Cover Quad Lower Left   Cloud Cover Quad Lower Right
Sun Elevation                 Sun Azimuth                   Scene Center Latitude
Scene Center Longitude        Corner Upper Left Latitude    Corner Upper Left Longitude
Corner Upper Right Latitude   Corner Upper Right Longitude  Corner Lower Left Latitude
Corner Lower Left Longitude   Corner Lower Right Latitude   Corner Lower Right Longitude
Browse  Exists                Data Category                 Map Projection LORa
Data Type L0Rp                Data Type Level 1             Elevation Source
Output Format                 Ephemeris Type                Corner UL Latitude Product
Corner UL Longitude Product   Corner UR Latitude Product    Corner UR Longitude Product
Corner LL Latitude Product    Corner LL Longitude Product   Corner LR Latitude Product
Corner LR Longitude Product   Reflective Samples            Reflective Lines
Thermal Lines                 Thermal Samples               Ground Control Points Model
Geometric RMSE Model          Geometric RMSE Model X        Geometric RMSE Model Y
Geometric RMSE Verify         Map Projection Level 1        Datum
Ellipsoid                     UTM Zone                      Vertical Long from Pole
True Scale Latitude           False Easting                 False Northing
Grid Cell Size Reflective     Grid Cell Size Thermal        Orientation
Resampling Option             Scene Center Latitude dec     Scene Center Longitude dec
Corner Upper Left Lat dec     Corner Upper Left Long dec    Corner Upper Right Lat dec
Corner Upper Right Long dec   Corner Lower Left Lat dec     Corner Lower Left Long dec
Corner Lower Right Lat dec    Corner Lower Right Long dec
    \end{Verbatim}

    \subsubsection{Extract IDs}

For right now, let's just extract the Landsat IDs from our search
result. We want to create a text file that only contains the Landsat IDs
we want to order with each ID separated by row.

You could do this step in a variety of ways - Excel, a text editor, R,
C, etc\ldots{} - but let's just use some utilities in Bash:

    \begin{Verbatim}[commandchars=\\\{\}]
{\color{incolor}In [{\color{incolor}53}]:} \PY{o}{\PYZpc{}\PYZpc{}}\PY{k}{bash}
         \PY{c}{\PYZsh{} First let\PYZsq{}s move to our working directory \PYZhy{} USE YOUR OWN DIRECTORY}
         \PY{n}{cd} \PY{o}{/}\PY{n}{projectnb}\PY{o}{/}\PY{n}{landsat}\PY{o}{/}\PY{n}{users}\PY{o}{/}\PY{n}{ceholden}\PY{o}{/}\PY{l+m+mi}{2014}\PY{n}{\PYZus{}Landsat\PYZus{}Preprocess}\PY{o}{/}
         
         \PY{c}{\PYZsh{} Let\PYZsq{}s test our command and only print the first few lines using \PYZdq{}head\PYZdq{}}
         \PY{n}{cat} \PY{l+m+mi}{1}\PY{n}{\PYZus{}EarthExplorer}\PY{o}{/}\PY{n}{LSR\PYZus{}LANDSAT\PYZus{}ETM\PYZus{}COMBINED\PYZus{}23722}\PY{o}{.}\PY{n}{txt} \PY{o}{|} \PY{n}{awk} \PY{o}{\PYZhy{}}\PY{n}{F} \PY{l+s}{\PYZsq{}}\PY{l+s}{,}\PY{l+s}{\PYZsq{}} \PY{l+s}{\PYZsq{}}\PY{l+s}{NR \PYZgt{} 1 \PYZob{} print \PYZdl{}2 \PYZcb{}}\PY{l+s}{\PYZsq{}} \PY{o}{|} \PY{n}{head} \PY{o}{\PYZhy{}}\PY{l+m+mi}{5}
\end{Verbatim}

    \begin{Verbatim}[commandchars=\\\{\}]
LE70200472014109ASN00
LE70200472014093ASN00
LE70200472014077ASN00
LE70200472014061EDC00
LE70200472014045EDC00
    \end{Verbatim}

    \begin{Verbatim}[commandchars=\\\{\}]
{\color{incolor}In [{\color{incolor}54}]:} \PY{o}{\PYZpc{}\PYZpc{}}\PY{k}{bash}
         \PY{c}{\PYZsh{} Now let\PYZsq{}s save our IDs to our output file}
         
         \PY{c}{\PYZsh{} My results are located within \PYZdq{}1\PYZus{}EarthExplorer\PYZdq{} and I will extract the IDs to a file named \PYZdq{}p020r047\PYZus{}submit.txt\PYZdq{}}
         \PY{n}{output}\PY{o}{=}\PY{l+m+mi}{2}\PY{n}{\PYZus{}ESPA}\PY{o}{/}\PY{n}{p020r047\PYZus{}submit}\PY{o}{.}\PY{n}{txt}
         
         \PY{c}{\PYZsh{} First I create the text file using the output from AWK for Landsat 7 ETM+ using \PYZdq{}\PYZgt{}\PYZdq{} to redirect from stdout to a file}
         \PY{n}{cat} \PY{l+m+mi}{1}\PY{n}{\PYZus{}EarthExplorer}\PY{o}{/}\PY{n}{LSR\PYZus{}LANDSAT\PYZus{}ETM\PYZus{}COMBINED\PYZus{}23722}\PY{o}{.}\PY{n}{txt} \PY{o}{|} \PY{n}{awk} \PY{o}{\PYZhy{}}\PY{n}{F} \PY{l+s}{\PYZsq{}}\PY{l+s}{,}\PY{l+s}{\PYZsq{}} \PY{l+s}{\PYZsq{}}\PY{l+s}{NR \PYZgt{} 1 \PYZob{} print \PYZdl{}2 \PYZcb{}}\PY{l+s}{\PYZsq{}} \PY{o}{\PYZgt{}} \PY{err}{\PYZdl{}}\PY{n}{output}
         \PY{c}{\PYZsh{} Now I append this text file with the IDs from Landsat 4\PYZhy{}5 TM using \PYZdq{}\PYZgt{}\PYZgt{}\PYZdq{} to append}
         \PY{n}{cat} \PY{l+m+mi}{1}\PY{n}{\PYZus{}EarthExplorer}\PY{o}{/}\PY{n}{LSR\PYZus{}LANDSAT\PYZus{}TM\PYZus{}23721}\PY{o}{.}\PY{n}{txt} \PY{o}{|} \PY{n}{awk} \PY{o}{\PYZhy{}}\PY{n}{F} \PY{l+s}{\PYZsq{}}\PY{l+s}{,}\PY{l+s}{\PYZsq{}} \PY{l+s}{\PYZsq{}}\PY{l+s}{NR \PYZgt{} 1 \PYZob{} print \PYZdl{}2 \PYZcb{}}\PY{l+s}{\PYZsq{}} \PY{o}{\PYZgt{}\PYZgt{}} \PY{err}{\PYZdl{}}\PY{n}{output}
\end{Verbatim}

    \begin{Verbatim}[commandchars=\\\{\}]
{\color{incolor}In [{\color{incolor}57}]:} \PY{o}{\PYZpc{}\PYZpc{}}\PY{k}{bash}
         \PY{c}{\PYZsh{} We can check to see how many images we have by concatenating our output file and piping it to \PYZdq{}wc\PYZdq{} (word count)}
         \PY{n}{n}\PY{o}{=}\PY{err}{\PYZdl{}}\PY{p}{(}\PY{n}{cat} \PY{l+m+mi}{2}\PY{n}{\PYZus{}ESPA}\PY{o}{/}\PY{n}{p020r047\PYZus{}submit}\PY{o}{.}\PY{n}{txt} \PY{o}{|} \PY{n}{wc} \PY{o}{\PYZhy{}}\PY{n}{l}\PY{p}{)}
         
         \PY{n}{echo} \PY{l+s}{\PYZdq{}}\PY{l+s}{You have \PYZdl{}n Landsat images in your text file}\PY{l+s}{\PYZdq{}}
\end{Verbatim}

    \begin{Verbatim}[commandchars=\\\{\}]
You have 438 Landsat images in your text file
    \end{Verbatim}

    \subsubsection{Submit text file of Landsat IDs to EROS Science
Processing Architecture (ESPA)}

In a web browser, navigate to:

https://espa.cr.usgs.gov

and login using your username and password (they're all defaults, so you
can also ask me for now, but it is important to register for tracking
purposes).

Once you log in, you'll be taken to the order page:

Enter your email address, click ``Browse'', and navigate to your list of
Landsat IDs.

You have many dataset options for your download, but it is always a good
idea to download the ``Source Metadata'' because this product contains
the Landsat ``MTL'' metadata file.

As pictured, I suggest downloading the following in addition to the
metadata:

\begin{itemize}
\itemsep1pt\parskip0pt\parsep0pt
\item
  ``Source Products''

  \begin{itemize}
  \itemsep1pt\parskip0pt\parsep0pt
  \item
    original TIF imagery in units of DNs
  \end{itemize}
\item
  ``Surface Reflectance''

  \begin{itemize}
  \itemsep1pt\parskip0pt\parsep0pt
  \item
    ``lndsr*hdf'' LEDAPS output, and the ``CFmask'' cloud mask
  \end{itemize}
\item
  ``Band 6 Brightness Temperature''

  \begin{itemize}
  \itemsep1pt\parskip0pt\parsep0pt
  \item
    ``lndth*hdf'' LEDAPS thermal brightness temperature
  \end{itemize}
\end{itemize}

I recommend downloading the ``Source Products'' even though the
``Surface Reflectance'' imagery has a Fmask cloud mask band because the
Landsat CDR version of Fmask is ran using default parameters (cloud
probability = 22.5, and cloud, shadow, and snow buffering of 3 pixels),
but you may require a more conservative mask depending on your scene
location or usage scenarios.

You also have the potential of having your output datasets resized,
reprojected, or subset.

When you have submitted your order, you will receive a confirmation
email and will be taken to your order status page:

    \subsection{Step 3. Download completed order onto GEO cluster}

When you receive an email update notifying you that your images have
been processed, you may access them here:

http://espa.cr.usgs.gov/status/

Simply input the email address associated with your order, select your
order, and you'll find a page with each image and it's status:

The ESPA interface provides you with a URL for downloading the image
archive and a checksum useful for assuring your download completed
without error (see http://en.wikipedia.org/wiki/Checksum).

Download onto the cluster

The quickest and easiest way of downloading these datasets onto the
cluster would be using Firefox's ``DownThemAll'' addon because it easily
allows for parallel downloads
(https://addons.mozilla.org/en-US/firefox/addon/downthemall/).

One simple way of downloading the datasets would be using ``wget''
(enter ``man wget'' for more information).

In this example, we use ``wget'' with several options: + ``-nc''
specifies ``no clobber'', or no overwrite + ``-r'' specifies recursion,
so we can find our files beneath the download URL + ``-A tar.gz,cksum''
is an accept list that specifies a pattern for files we will download.
Files not matching the pattern are ignored + ``-nd'' will prevent
downloading directories + ``-nv'' will turn off verbose

    \begin{Verbatim}[commandchars=\\\{\}]
{\color{incolor}In [{\color{incolor}3}]:} \PY{o}{\PYZpc{}\PYZpc{}}\PY{k}{bash}
        \PY{n}{cd} \PY{o}{/}\PY{n}{projectnb}\PY{o}{/}\PY{n}{landsat}\PY{o}{/}\PY{n}{users}\PY{o}{/}\PY{n}{ceholden}\PY{o}{/}\PY{l+m+mi}{2014}\PY{n}{\PYZus{}Landsat\PYZus{}Preprocess}\PY{o}{/}\PY{l+m+mi}{3}\PY{n}{\PYZus{}Download}
        
        \PY{n}{url}\PY{o}{=}\PY{n}{http}\PY{p}{:}\PY{o}{/}\PY{o}{/}\PY{n}{espa}\PY{o}{.}\PY{n}{cr}\PY{o}{.}\PY{n}{usgs}\PY{o}{.}\PY{n}{gov}\PY{o}{/}\PY{n}{status}\PY{o}{/}\PY{n}{ceholden}\PY{o}{\PYZpc{}}\PY{k}{40bu}\PY{o}{.}\PY{n}{edu}\PY{o}{\PYZhy{}}\PY{l+m+mi}{572014}\PY{o}{\PYZhy{}}\PY{l+m+mi}{17955}\PY{o}{/}
        
        \PY{n}{wget} \PY{o}{\PYZhy{}}\PY{n}{nc} \PY{o}{\PYZhy{}}\PY{n}{r} \PY{o}{\PYZhy{}}\PY{n}{A} \PY{n}{tar}\PY{o}{.}\PY{n}{gz}\PY{p}{,}\PY{n}{cksum} \PY{o}{\PYZhy{}}\PY{n}{nd} \PY{o}{\PYZhy{}}\PY{n}{nv} \PY{err}{\PYZdl{}}\PY{n}{url}
        
        \PY{n}{echo} \PY{l+s}{\PYZdq{}}\PY{l+s}{Done!}\PY{l+s}{\PYZdq{}}
\end{Verbatim}

    \begin{Verbatim}[commandchars=\\\{\}]
Done!
    \end{Verbatim}

    Validate checksums of downloads

It is important, but not always necessary, to assert that the files
stored on the USGS's servers are fully intact once downloaded to our
cluster. We can assert that the files are in tact with high confidence
by checking to make sure the ``checksum'' of the file matches once
downloaded to BU.

For each archive you download, there will be a corresponding ``*.cksum''
file. The checksum provided is output from the Unix ``cksum'' command
(http://en.wikipedia.org/wiki/Cksum). Unfortuantely, there isn't a
built-in mechanism in this program to validate checksums, so we'll do
this in a Bash loop and if statement.

    \begin{Verbatim}[commandchars=\\\{\}]
{\color{incolor}In [{\color{incolor}18}]:} \PY{o}{\PYZpc{}\PYZpc{}}\PY{k}{bash}
         \PY{n}{cd} \PY{o}{/}\PY{n}{projectnb}\PY{o}{/}\PY{n}{landsat}\PY{o}{/}\PY{n}{users}\PY{o}{/}\PY{n}{ceholden}\PY{o}{/}\PY{l+m+mi}{2014}\PY{n}{\PYZus{}Landsat\PYZus{}Preprocess}\PY{o}{/}\PY{l+m+mi}{3}\PY{n}{\PYZus{}Download}
         
         \PY{c}{\PYZsh{} Loop over cksum files}
         \PY{k}{for} \PY{n}{checksum} \PY{o+ow}{in} \PY{err}{\PYZdl{}}\PY{p}{(}\PY{n}{find} \PY{o}{.}\PY{o}{/} \PY{o}{\PYZhy{}}\PY{n}{name} \PY{l+s}{\PYZsq{}}\PY{l+s}{L*.cksum}\PY{l+s}{\PYZsq{}}\PY{p}{)}\PY{p}{;} \PY{n}{do}
             \PY{c}{\PYZsh{} Find basename of file and remove extension to match up with archive}
             \PY{n}{bn}\PY{o}{=}\PY{err}{\PYZdl{}}\PY{p}{(}\PY{n}{basename} \PY{err}{\PYZdl{}}\PY{n}{checksum} \PY{o}{|} \PY{n}{awk} \PY{o}{\PYZhy{}}\PY{n}{F} \PY{l+s}{\PYZsq{}}\PY{l+s}{.}\PY{l+s}{\PYZsq{}} \PY{l+s}{\PYZsq{}}\PY{l+s}{\PYZob{} print \PYZdl{}1 \PYZcb{}}\PY{l+s}{\PYZsq{}}\PY{p}{)}
             
             \PY{c}{\PYZsh{} Test to see if archive exists}
             \PY{n}{archive}\PY{o}{=}\PY{err}{\PYZdl{}}\PY{p}{\PYZob{}}\PY{n}{bn}\PY{p}{\PYZcb{}}\PY{o}{.}\PY{n}{tar}\PY{o}{.}\PY{n}{gz}
             \PY{k}{if} \PY{p}{[} \PY{err}{!} \PY{o}{\PYZhy{}}\PY{n}{f} \PY{err}{\PYZdl{}}\PY{n}{archive} \PY{p}{]}\PY{p}{;} \PY{n}{then}
                 \PY{n}{echo} \PY{l+s}{\PYZdq{}}\PY{l+s}{\PYZdl{}bn has no matching archive}\PY{l+s}{\PYZdq{}}
                 \PY{k}{continue}
             \PY{n}{fi}
             
             \PY{c}{\PYZsh{} If archive exists, then validate checksum}
             \PY{n}{test}\PY{o}{=}\PY{err}{\PYZdl{}}\PY{p}{(}\PY{n}{cksum} \PY{err}{\PYZdl{}}\PY{n}{archive}\PY{p}{)}
             \PY{k}{if} \PY{p}{[} \PY{l+s}{\PYZdq{}}\PY{l+s}{\PYZdl{}test}\PY{l+s}{\PYZdq{}} \PY{o}{!=} \PY{l+s}{\PYZdq{}}\PY{l+s}{\PYZdl{}(cat \PYZdl{}checksum)}\PY{l+s}{\PYZdq{}} \PY{p}{]}\PY{p}{;} \PY{n}{then}
                 \PY{n}{echo} \PY{l+s}{\PYZdq{}}\PY{l+s}{!!!!! WARNING !!!!!}\PY{l+s}{\PYZdq{}}
                 \PY{n}{echo} \PY{l+s}{\PYZdq{}}\PY{l+s}{\PYZdl{}bn may be corrupted}\PY{l+s}{\PYZdq{}}
                 \PY{n}{echo} \PY{l+s}{\PYZdq{}}\PY{l+s}{!!!!! WARNING !!!!!}\PY{l+s}{\PYZdq{}}
             \PY{k}{else}
                 \PY{n}{echo} \PY{l+s}{\PYZdq{}}\PY{l+s}{\PYZdl{}bn is OK}\PY{l+s}{\PYZdq{}}
             \PY{n}{fi}
         \PY{n}{done}
\end{Verbatim}

    \begin{Verbatim}[commandchars=\\\{\}]
LE70200472014093-SC20140507171414 is OK
LE70200472014109-SC20140507171414 is OK
LE70200472014077-SC20140507171413 is OK
LE70200472014061-SC20140507171414 is OK
    \end{Verbatim}

    \subsection{Step 4. Organize and extract imagery from archives}

If you have any prior experience with Landsat imagery, you'll notice
that the archives we downloaded have an unusual filename:

\begin{verbatim}
                LE70200472011261-SC20140507171414.tar.gz
                
\end{verbatim}

The first half of this string is the Landsat ID, minus the receiving
station information. The second half of this string details the timing
of when the Landsat image was processed through ESPA.

Since we're not very interested in the processing detail, we will
extract each of these archives to a folder based on the reference ID of
the Landsat image.

    \begin{Verbatim}[commandchars=\\\{\}]
{\color{incolor}In [{\color{incolor}21}]:} \PY{o}{\PYZpc{}\PYZpc{}}\PY{k}{bash}
         \PY{n}{cd} \PY{o}{/}\PY{n}{projectnb}\PY{o}{/}\PY{n}{landsat}\PY{o}{/}\PY{n}{users}\PY{o}{/}\PY{n}{ceholden}\PY{o}{/}\PY{l+m+mi}{2014}\PY{n}{\PYZus{}Landsat\PYZus{}Preprocess}\PY{o}{/}\PY{l+m+mi}{4}\PY{n}{\PYZus{}Organize}
         
         \PY{c}{\PYZsh{} We want to find things ONLY in our current directory, not in any subfolders}
         \PY{c}{\PYZsh{}     So, we use \PYZhy{}maxdepth 1 option}
         \PY{c}{\PYZsh{}     You could also just use \PYZdq{}ls *tar.gz\PYZdq{}, }
         \PY{c}{\PYZsh{}     but find is good to know because it gives you a lot of control}
         \PY{n}{n}\PY{o}{=}\PY{err}{\PYZdl{}}\PY{p}{(}\PY{n}{find} \PY{o}{.}\PY{o}{/} \PY{o}{\PYZhy{}}\PY{n}{maxdepth} \PY{l+m+mi}{1} \PY{o}{\PYZhy{}}\PY{n}{name} \PY{l+s}{\PYZsq{}}\PY{l+s}{*tar.gz}\PY{l+s}{\PYZsq{}} \PY{o}{|} \PY{n}{wc} \PY{o}{\PYZhy{}}\PY{n}{l}\PY{p}{)}
         \PY{n}{i}\PY{o}{=}\PY{l+m+mi}{1}
         
         \PY{k}{for} \PY{n}{archive} \PY{o+ow}{in} \PY{err}{\PYZdl{}}\PY{p}{(}\PY{n}{find} \PY{o}{.}\PY{o}{/} \PY{o}{\PYZhy{}}\PY{n}{maxdepth} \PY{l+m+mi}{1} \PY{o}{\PYZhy{}}\PY{n}{name} \PY{l+s}{\PYZsq{}}\PY{l+s}{*tar.gz}\PY{l+s}{\PYZsq{}}\PY{p}{)}\PY{p}{;} \PY{n}{do}
             \PY{n}{echo} \PY{l+s}{\PYZdq{}}\PY{l+s}{\PYZlt{}\PYZhy{}\PYZhy{}\PYZhy{}\PYZhy{}\PYZhy{} \PYZdl{}i / \PYZdl{}n: \PYZdl{}(basename \PYZdl{}img)}\PY{l+s}{\PYZdq{}}
             
             \PY{c}{\PYZsh{} Create temporary folder for storage}
             \PY{n}{mkdir} \PY{n}{temp}
             
             \PY{c}{\PYZsh{} Extract archive to temporary folder}
             \PY{n}{tar} \PY{o}{\PYZhy{}}\PY{n}{xzvf} \PY{err}{\PYZdl{}}\PY{n}{archive} \PY{o}{\PYZhy{}}\PY{n}{C} \PY{n}{temp}\PY{o}{/}
             
             \PY{c}{\PYZsh{} Find ID based on MTL file\PYZsq{}s filename}
             \PY{n}{mtl}\PY{o}{=}\PY{err}{\PYZdl{}}\PY{p}{(}\PY{n}{find} \PY{n}{temp}\PY{o}{/} \PY{o}{\PYZhy{}}\PY{n}{name} \PY{l+s}{\PYZsq{}}\PY{l+s}{L*MTL.txt}\PY{l+s}{\PYZsq{}}\PY{p}{)}
             
             \PY{c}{\PYZsh{} Test to make sure we found it}
             \PY{k}{if} \PY{p}{[} \PY{err}{!} \PY{o}{\PYZhy{}}\PY{n}{f} \PY{err}{\PYZdl{}}\PY{n}{mtl} \PY{p}{]}\PY{p}{;} \PY{n}{then}
                 \PY{n}{echo} \PY{l+s}{\PYZdq{}}\PY{l+s}{Could not find MTL file for \PYZdl{}archive}\PY{l+s}{\PYZdq{}}
                 \PY{k}{break}
             \PY{n}{fi}
             
             \PY{c}{\PYZsh{} Use AWK to remove \PYZus{}MTL.txt}
             \PY{n+nb}{id}\PY{o}{=}\PY{err}{\PYZdl{}}\PY{p}{(}\PY{n}{basename} \PY{err}{\PYZdl{}}\PY{n}{mtl} \PY{o}{|} \PY{n}{awk} \PY{o}{\PYZhy{}}\PY{n}{F} \PY{l+s}{\PYZsq{}}\PY{l+s}{\PYZus{}}\PY{l+s}{\PYZsq{}} \PY{l+s}{\PYZsq{}}\PY{l+s}{\PYZob{} print \PYZdl{}1 \PYZcb{}}\PY{l+s}{\PYZsq{}}\PY{p}{)}
             
             \PY{c}{\PYZsh{} Move archive into temporary folder}
             \PY{n}{mv} \PY{err}{\PYZdl{}}\PY{n}{archive} \PY{n}{temp}\PY{o}{/}
             
             \PY{c}{\PYZsh{} Rename archive}
             \PY{n}{mv} \PY{n}{temp} \PY{err}{\PYZdl{}}\PY{n+nb}{id}
             
             \PY{c}{\PYZsh{} Iterate count}
             \PY{n}{let} \PY{n}{i}\PY{o}{+}\PY{o}{=}\PY{l+m+mi}{1}
         \PY{n}{done}
         
         \PY{n}{echo} \PY{l+s}{\PYZdq{}}\PY{l+s}{Done!}\PY{l+s}{\PYZdq{}}
\end{Verbatim}

    \begin{Verbatim}[commandchars=\\\{\}]
Done!
    \end{Verbatim}

    Filter L1G images

If you recall when we filtered our dataset from the Landsat archive, we
had selected the ``Processing Required'' option. This means that some of
the images we downloaded that had this status will be L1T, but some of
them will be L1G (not orthorectified).

In order to remove the L1G images, we can run through our image stacks
and do a quick ``grep'' on each MTL file:

    \begin{Verbatim}[commandchars=\\\{\}]
{\color{incolor}In [{\color{incolor}83}]:} \PY{o}{\PYZpc{}\PYZpc{}}\PY{k}{bash}
         \PY{n}{cd} \PY{o}{/}\PY{n}{projectnb}\PY{o}{/}\PY{n}{landsat}\PY{o}{/}\PY{n}{users}\PY{o}{/}\PY{n}{ceholden}\PY{o}{/}\PY{l+m+mi}{2014}\PY{n}{\PYZus{}Landsat\PYZus{}Preprocess}\PY{o}{/}\PY{l+m+mi}{4}\PY{n}{\PYZus{}Organize}
         
         \PY{k}{if} \PY{p}{[} \PY{err}{!} \PY{o}{\PYZhy{}}\PY{n}{d} \PY{n}{L1G} \PY{p}{]}\PY{p}{;} \PY{n}{then}
             \PY{n}{mkdir} \PY{n}{L1G}\PY{o}{/}
         \PY{n}{fi}
         
         \PY{k}{for} \PY{n}{mtl} \PY{o+ow}{in} \PY{err}{\PYZdl{}}\PY{p}{(}\PY{n}{find} \PY{o}{.}\PY{o}{/} \PY{o}{\PYZhy{}}\PY{n}{name} \PY{l+s}{\PYZsq{}}\PY{l+s}{L*MTL.txt}\PY{l+s}{\PYZsq{}}\PY{p}{)}\PY{p}{;} \PY{n}{do}
             \PY{n+nb}{id}\PY{o}{=}\PY{err}{\PYZdl{}}\PY{p}{(}\PY{n}{basename} \PY{err}{\PYZdl{}}\PY{p}{(}\PY{n}{dirname} \PY{err}{\PYZdl{}}\PY{n}{mtl}\PY{p}{)}\PY{p}{)}
             
             \PY{n}{l1t}\PY{o}{=}\PY{err}{\PYZdl{}}\PY{p}{(}\PY{n}{grep} \PY{l+s}{\PYZdq{}}\PY{l+s}{L1T}\PY{l+s}{\PYZdq{}} \PY{err}{\PYZdl{}}\PY{n}{mtl}\PY{p}{)}
             
             \PY{k}{if} \PY{p}{[} \PY{l+s}{\PYZdq{}}\PY{l+s}{\PYZdl{}l1t}\PY{l+s}{\PYZdq{}} \PY{o}{==} \PY{l+s}{\PYZdq{}}\PY{l+s}{\PYZdq{}} \PY{p}{]}\PY{p}{;} \PY{n}{then}
                 \PY{n}{echo} \PY{l+s}{\PYZdq{}}\PY{l+s}{\PYZdl{}id is not L1T}\PY{l+s}{\PYZdq{}}
                 \PY{n}{mv} \PY{err}{\PYZdl{}}\PY{p}{(}\PY{n}{dirname} \PY{err}{\PYZdl{}}\PY{n}{mtl}\PY{p}{)} \PY{n}{L1G}
             \PY{k}{else}
                 \PY{n}{echo} \PY{l+s}{\PYZdq{}}\PY{l+s}{\PYZdl{}id is L1T}\PY{l+s}{\PYZdq{}}
             \PY{n}{fi}
             
         \PY{n}{done}
\end{Verbatim}

    \begin{Verbatim}[commandchars=\\\{\}]
LE70200472014061EDC00 is L1T
LE70200472014077ASN00 is L1T
LE70200472014093ASN00 is L1T
LE70200472014109ASN00 is L1T
    \end{Verbatim}

    \begin{Verbatim}[commandchars=\\\{\}]
mkdir: cannot create directory `L1G/': File exists
    \end{Verbatim}

    \subsection{Step 5. Combine dataset bands into ``stacked'' images of
common geographic extent}

We now have the images organized by their Landsat ID and extracted from
the archives. The next step in our workflow is to create ``layer
stacks'' - or multiband images - that contain the bands we want to use
in our analysis. For example, we might want to create a multiband image
for each Landsat ID that contains all Landsat bands, ordered by
wavelength, and the Fmask result.

This workflow can be accomplished using a variety of software - ENVI,
ArcGIS, ERDAS, R, and QGIS, for example - but it would be enormously
time consuming to use these programs to accomplish our goal for hundreds
of images. Instead, we will use the free program
\textbf{\emph{gdal\_merge.py}} from the GDAL command line set of
programs (http://www.gdal.org/ or http://www.gdal.org/gdal\_merge.html).

One complicating factor in this workflow is that our LEDAPS output are
currently stored as HDF files which stores bands as ``subdatasets''
instead of bands, making referencing them a little difficult (see
http://www.gdal.org/frmt\_hdf4.html).

Sometime soon, the ESPA processing system will allow you to download
your datasets as either HDF, GeoTIFF, or binary files
(http://landsat.usgs.gov/sr\_samples.php).

First we will begin my loading the module for GDAL and inspecting one of
the LEDAPS outputs.

    \begin{Verbatim}[commandchars=\\\{\}]
{\color{incolor}In [{\color{incolor}45}]:} \PY{o}{\PYZpc{}\PYZpc{}}\PY{k}{bash}
         \PY{n}{module} \PY{n}{purge}
         \PY{n}{module} \PY{n}{load} \PY{n}{gdal}\PY{o}{/}\PY{l+m+mf}{1.10}\PY{o}{.}\PY{l+m+mi}{0}
         
         \PY{n}{cd} \PY{o}{/}\PY{n}{projectnb}\PY{o}{/}\PY{n}{landsat}\PY{o}{/}\PY{n}{users}\PY{o}{/}\PY{n}{ceholden}\PY{o}{/}\PY{l+m+mi}{2014}\PY{n}{\PYZus{}Landsat\PYZus{}Preprocess}\PY{o}{/}\PY{l+m+mi}{5}\PY{n}{\PYZus{}Stack}\PY{o}{/}\PY{n}{LE70200472014061EDC00}
         
         \PY{n}{gdalinfo} \PY{n}{lndsr}\PY{o}{.}\PY{n}{LE70200472014061EDC00}\PY{o}{.}\PY{n}{hdf}
\end{Verbatim}

    \begin{Verbatim}[commandchars=\\\{\}]
Driver: HDF4/Hierarchical Data Format Release 4
Files: lndsr.LE70200472014061EDC00.hdf
Size is 512, 512
Coordinate System is `'
Metadata:
  AcquisitionDate=2014-03-02T16:19:41.029800Z
  CFmaskVersion=1.3.0
  Cloud Mask Algo Version=CMReflectanceBasedv1.0
  DataProvider=USGS/EROS
  EastBoundingCoordinate=-88.6776995141039
  HDFEOSVersion=4.2
  HDFVersion=4.2.5
  Instrument=ETM
  LEDAPSVersion=1.3.1
  Level1ProductionDate=2014-03-02T00:00:00.000000Z
  LocalGranuleID=L7ESR.a2014061.w2p020r047.020.2014127222256.hdf
  LowerRightCornerLatLong=17.8410695070442, -88.6778422897102
  LPGSMetadataFile=LE70200472014061EDC00\_MTL.txt
  NorthBoundingCoordinate=19.7654335783953
  OrientationAngle=0
  PixelSize=30
  ProductionDate=2014-05-07T22:22:56Z
  ReflBias=-6.97874021530151, -7.19881916046143, -5.62165403366089, -6.06929111480713, -1.12621998786926, -0.393898010253906
  ReflGains=0.778739988803864, 0.798819005489349, 0.621653020381927, 0.969290971755981, 0.126220002770424, 0.0438980013132095
  Satellite=LANDSAT\_7
  ShortName=L7ESR
  SolarAzimuth=130.5961761
  SolarZenith=37.7718544
  SouthBoundingCoordinate=17.8077011670647
  ThermalBias=-0.0670870020985603
  ThermalGain=0.0670870020985603
  UpperLeftCornerLatLong=19.7282076307976, -91.0190580192774
  WestBoundingCoordinate=-91.0192042045475
  WRS\_Path=20
  WRS\_Row=47
  WRS\_System=2
Subdatasets:
  SUBDATASET\_1\_NAME=HDF4\_EOS:EOS\_GRID:"lndsr.LE70200472014061EDC00.hdf":Grid:band1
  SUBDATASET\_1\_DESC=[7101x8121] band1 Grid (16-bit integer)
  SUBDATASET\_2\_NAME=HDF4\_EOS:EOS\_GRID:"lndsr.LE70200472014061EDC00.hdf":Grid:band2
  SUBDATASET\_2\_DESC=[7101x8121] band2 Grid (16-bit integer)
  SUBDATASET\_3\_NAME=HDF4\_EOS:EOS\_GRID:"lndsr.LE70200472014061EDC00.hdf":Grid:band3
  SUBDATASET\_3\_DESC=[7101x8121] band3 Grid (16-bit integer)
  SUBDATASET\_4\_NAME=HDF4\_EOS:EOS\_GRID:"lndsr.LE70200472014061EDC00.hdf":Grid:band4
  SUBDATASET\_4\_DESC=[7101x8121] band4 Grid (16-bit integer)
  SUBDATASET\_5\_NAME=HDF4\_EOS:EOS\_GRID:"lndsr.LE70200472014061EDC00.hdf":Grid:band5
  SUBDATASET\_5\_DESC=[7101x8121] band5 Grid (16-bit integer)
  SUBDATASET\_6\_NAME=HDF4\_EOS:EOS\_GRID:"lndsr.LE70200472014061EDC00.hdf":Grid:band7
  SUBDATASET\_6\_DESC=[7101x8121] band7 Grid (16-bit integer)
  SUBDATASET\_7\_NAME=HDF4\_EOS:EOS\_GRID:"lndsr.LE70200472014061EDC00.hdf":Grid:atmos\_opacity
  SUBDATASET\_7\_DESC=[7101x8121] atmos\_opacity Grid (16-bit integer)
  SUBDATASET\_8\_NAME=HDF4\_EOS:EOS\_GRID:"lndsr.LE70200472014061EDC00.hdf":Grid:fill\_QA
  SUBDATASET\_8\_DESC=[7101x8121] fill\_QA Grid (8-bit unsigned integer)
  SUBDATASET\_9\_NAME=HDF4\_EOS:EOS\_GRID:"lndsr.LE70200472014061EDC00.hdf":Grid:DDV\_QA
  SUBDATASET\_9\_DESC=[7101x8121] DDV\_QA Grid (8-bit unsigned integer)
  SUBDATASET\_10\_NAME=HDF4\_EOS:EOS\_GRID:"lndsr.LE70200472014061EDC00.hdf":Grid:cloud\_QA
  SUBDATASET\_10\_DESC=[7101x8121] cloud\_QA Grid (8-bit unsigned integer)
  SUBDATASET\_11\_NAME=HDF4\_EOS:EOS\_GRID:"lndsr.LE70200472014061EDC00.hdf":Grid:cloud\_shadow\_QA
  SUBDATASET\_11\_DESC=[7101x8121] cloud\_shadow\_QA Grid (8-bit unsigned integer)
  SUBDATASET\_12\_NAME=HDF4\_EOS:EOS\_GRID:"lndsr.LE70200472014061EDC00.hdf":Grid:snow\_QA
  SUBDATASET\_12\_DESC=[7101x8121] snow\_QA Grid (8-bit unsigned integer)
  SUBDATASET\_13\_NAME=HDF4\_EOS:EOS\_GRID:"lndsr.LE70200472014061EDC00.hdf":Grid:land\_water\_QA
  SUBDATASET\_13\_DESC=[7101x8121] land\_water\_QA Grid (8-bit unsigned integer)
  SUBDATASET\_14\_NAME=HDF4\_EOS:EOS\_GRID:"lndsr.LE70200472014061EDC00.hdf":Grid:adjacent\_cloud\_QA
  SUBDATASET\_14\_DESC=[7101x8121] adjacent\_cloud\_QA Grid (8-bit unsigned integer)
  SUBDATASET\_15\_NAME=HDF4\_EOS:EOS\_GRID:"lndsr.LE70200472014061EDC00.hdf":Grid:band6
  SUBDATASET\_15\_DESC=[7101x8121] band6 Grid (16-bit integer)
  SUBDATASET\_16\_NAME=HDF4\_EOS:EOS\_GRID:"lndsr.LE70200472014061EDC00.hdf":Grid:band6\_fill\_QA
  SUBDATASET\_16\_DESC=[7101x8121] band6\_fill\_QA Grid (8-bit unsigned integer)
  SUBDATASET\_17\_NAME=HDF4\_EOS:EOS\_GRID:"lndsr.LE70200472014061EDC00.hdf":Grid:fmask\_band
  SUBDATASET\_17\_DESC=[7101x8121] fmask\_band Grid (8-bit unsigned integer)
Corner Coordinates:
Upper Left  (    0.0,    0.0)
Lower Left  (    0.0,  512.0)
Upper Right (  512.0,    0.0)
Lower Right (  512.0,  512.0)
Center      (  256.0,  256.0)
    \end{Verbatim}

    The LEDAPS surface reflectance HDF file contains 17 subdatasets which
store the surface reflectance bands, a variety of QA/QC bands, the
thermal brightness temperature band, and the Fmask band.

In order to reference an individual subdataset from this file, we have
to reference the name of the subdataset directly. For example,
``HDF4\_EOS:EOS\_GRID:''lndsr.LE70200472014061EDC00.hdf``:Grid:fmask\_band''.

    \begin{Verbatim}[commandchars=\\\{\}]
{\color{incolor}In [{\color{incolor}46}]:} \PY{o}{\PYZpc{}\PYZpc{}}\PY{k}{bash}
         \PY{n}{cd} \PY{o}{/}\PY{n}{projectnb}\PY{o}{/}\PY{n}{landsat}\PY{o}{/}\PY{n}{users}\PY{o}{/}\PY{n}{ceholden}\PY{o}{/}\PY{l+m+mi}{2014}\PY{n}{\PYZus{}Landsat\PYZus{}Preprocess}\PY{o}{/}\PY{l+m+mi}{5}\PY{n}{\PYZus{}Stack}\PY{o}{/}\PY{n}{LE70200472014061EDC00}
         
         \PY{n}{gdalinfo} \PY{n}{HDF4\PYZus{}EOS}\PY{p}{:}\PY{n}{EOS\PYZus{}GRID}\PY{p}{:}\PY{l+s}{\PYZdq{}}\PY{l+s}{lndsr.LE70200472014061EDC00.hdf}\PY{l+s}{\PYZdq{}}\PY{p}{:}\PY{n}{Grid}\PY{p}{:}\PY{n}{fmask\PYZus{}band}
\end{Verbatim}

    \begin{Verbatim}[commandchars=\\\{\}]
Driver: HDF4Image/HDF4 Dataset
Files: lndsr.LE70200472014061EDC00.hdf
Size is 8121, 7101
Coordinate System is:
PROJCS["UTM Zone 16, Northern Hemisphere",
    GEOGCS["Unknown datum based upon the WGS 84 ellipsoid",
        DATUM["Not specified (based on WGS 84 spheroid)",
            SPHEROID["WGS 84",6378137,298.257223563,
                AUTHORITY["EPSG","7030"]]],
        PRIMEM["Greenwich",0],
        UNIT["degree",0.0174532925199433]],
    PROJECTION["Transverse\_Mercator"],
    PARAMETER["latitude\_of\_origin",0],
    PARAMETER["central\_meridian",-87],
    PARAMETER["scale\_factor",0.9996],
    PARAMETER["false\_easting",500000],
    PARAMETER["false\_northing",0],
    UNIT["Meter",1]]
Origin = (78585.000000000000000,2186415.000000000000000)
Pixel Size = (30.000000000000000,-30.000000000000000)
Metadata:
  \_FillValue=255
  AcquisitionDate=2014-03-02T16:19:41.029800Z
  CFmaskVersion=1.3.0
  Cloud Mask Algo Version=CMReflectanceBasedv1.0
  DataProvider=USGS/EROS
  EastBoundingCoordinate=-88.6776995141039
  HDFEOSVersion=4.2
  HDFVersion=4.2.5
  Instrument=ETM
  LEDAPSVersion=1.3.1
  Level1ProductionDate=2014-03-02T00:00:00.000000Z
  LocalGranuleID=L7ESR.a2014061.w2p020r047.020.2014127222256.hdf
  long\_name=fmask\_band
  LowerRightCornerLatLong=17.8410695070442, -88.6778422897102
  LPGSMetadataFile=LE70200472014061EDC00\_MTL.txt
  mask\_index=0 clear; 1 water; 2 cloud\_shadow; 3 snow; 4 cloud
  NorthBoundingCoordinate=19.7654335783953
  OrientationAngle=0
  PixelSize=30
  ProductionDate=2014-05-07T22:22:56Z
  ReflBias=-6.97874021530151, -7.19881916046143, -5.62165403366089, -6.06929111480713, -1.12621998786926, -0.393898010253906
  ReflGains=0.778739988803864, 0.798819005489349, 0.621653020381927, 0.969290971755981, 0.126220002770424, 0.0438980013132095
  Satellite=LANDSAT\_7
  ShortName=L7ESR
  SolarAzimuth=130.5961761
  SolarZenith=37.7718544
  SouthBoundingCoordinate=17.8077011670647
  ThermalBias=-0.0670870020985603
  ThermalGain=0.0670870020985603
  UpperLeftCornerLatLong=19.7282076307976, -91.0190580192774
  valid\_range=0, 4
  WestBoundingCoordinate=-91.0192042045475
  WRS\_Path=20
  WRS\_Row=47
  WRS\_System=2
Corner Coordinates:
Upper Left  (   78585.000, 2186415.000) ( 91d 1' 9.14"W, 19d43'42.02"N)
Lower Left  (   78585.000, 1973385.000) ( 90d58'25.95"W, 17d48'27.72"N)
Upper Right (  322215.000, 2186415.000) ( 88d41'48.80"W, 19d45'55.56"N)
Lower Right (  322215.000, 1973385.000) ( 88d40'39.72"W, 17d50'27.37"N)
Center      (  200400.000, 2079900.000) ( 89d50'31.44"W, 18d47'21.19"N)
Band 1 Block=8121x123 Type=Byte, ColorInterp=Gray
  Description = fmask\_band
  NoData Value=255
    \end{Verbatim}

    One thing to note is that the HDF files the USGS provides do not
properly describe the datum of the datasets.

\begin{verbatim}
                GEOGCS["Unknown datum based upon the WGS 84 ellipsoid",
                
\end{verbatim}

From the ``Landsat Dictionary''
(https://lta.cr.usgs.gov/landsat\_dictionary.html\#ellipsoid), we know
that Landsat data is using the WGS-84 ellipsoid.

Fortunately, we can overwrite the projection with the correct one by
either using the ``proj4'' string from one of the TIF files:

    \begin{Verbatim}[commandchars=\\\{\}]
{\color{incolor}In [{\color{incolor}48}]:} \PY{o}{\PYZpc{}\PYZpc{}}\PY{k}{bash}
         \PY{n}{cd} \PY{o}{/}\PY{n}{projectnb}\PY{o}{/}\PY{n}{landsat}\PY{o}{/}\PY{n}{users}\PY{o}{/}\PY{n}{ceholden}\PY{o}{/}\PY{l+m+mi}{2014}\PY{n}{\PYZus{}Landsat\PYZus{}Preprocess}\PY{o}{/}\PY{l+m+mi}{5}\PY{n}{\PYZus{}Stack}\PY{o}{/}\PY{n}{LE70200472014061EDC00}
         
         \PY{n}{gdalinfo} \PY{o}{\PYZhy{}}\PY{n}{proj4} \PY{n}{LE70200472014061EDC00\PYZus{}B1}\PY{o}{.}\PY{n}{TIF} \PY{o}{|} \PY{n}{grep} \PY{l+s}{\PYZdq{}}\PY{l+s}{+proj}\PY{l+s}{\PYZdq{}}
\end{Verbatim}

    \begin{Verbatim}[commandchars=\\\{\}]
'+proj=utm +zone=16 +datum=WGS84 +units=m +no\_defs '
    \end{Verbatim}

    Stacked images

To create a ``layer stack'' from this LEDAPS output, we first need to
identify which bands we would like to use and which subdatasets the
bands are contained in.

For a ``layer stack'' of the optical bands, the thermal band, and the
Fmask band, we would reference the first 6 subdatasets, the 15th, and
the 17th. Since we know ``gdalinfo'' will provide us with the correct
references to the subdatasets, we can loop over these subdataset indices
and store the names:

    \begin{Verbatim}[commandchars=\\\{\}]
{\color{incolor}In [{\color{incolor}38}]:} \PY{o}{\PYZpc{}\PYZpc{}}\PY{k}{bash}
         \PY{n}{cd} \PY{o}{/}\PY{n}{projectnb}\PY{o}{/}\PY{n}{landsat}\PY{o}{/}\PY{n}{users}\PY{o}{/}\PY{n}{ceholden}\PY{o}{/}\PY{l+m+mi}{2014}\PY{n}{\PYZus{}Landsat\PYZus{}Preprocess}\PY{o}{/}\PY{l+m+mi}{5}\PY{n}{\PYZus{}Stack}\PY{o}{/}\PY{n}{LE70200472014061EDC00}
         
         \PY{n}{image}\PY{o}{=}\PY{n}{lndsr}\PY{o}{.}\PY{n}{LE70200472014061EDC00}\PY{o}{.}\PY{n}{hdf}
         
         \PY{n}{bands}\PY{o}{=}\PY{l+s}{\PYZdq{}}\PY{l+s}{1 2 3 4 5 6 15 17}\PY{l+s}{\PYZdq{}}
         
         \PY{n}{sds\PYZus{}names}\PY{o}{=}\PY{l+s}{\PYZdq{}}\PY{l+s}{\PYZdq{}}
         
         \PY{k}{for} \PY{n}{b} \PY{o+ow}{in} \PY{err}{\PYZdl{}}\PY{n}{bands}\PY{p}{;} \PY{n}{do}
             \PY{n}{gdalinfo} \PY{err}{\PYZdl{}}\PY{n}{image} \PY{o}{|} \PY{n}{grep} \PY{l+s}{\PYZdq{}}\PY{l+s}{SUBDATASET\PYZus{}\PYZdl{}\PYZob{}b\PYZcb{}\PYZus{}NAME}\PY{l+s}{\PYZdq{}}
             \PY{n}{sds}\PY{o}{=}\PY{err}{\PYZdl{}}\PY{p}{(}\PY{n}{gdalinfo} \PY{err}{\PYZdl{}}\PY{n}{image} \PY{o}{|} \PY{n}{grep} \PY{l+s}{\PYZdq{}}\PY{l+s}{SUBDATASET\PYZus{}\PYZdl{}\PYZob{}b\PYZcb{}\PYZus{}NAME}\PY{l+s}{\PYZdq{}} \PY{o}{|} \PY{n}{awk} \PY{o}{\PYZhy{}}\PY{n}{F} \PY{l+s}{\PYZsq{}}\PY{l+s}{=}\PY{l+s}{\PYZsq{}} \PY{l+s}{\PYZsq{}}\PY{l+s}{\PYZob{} print \PYZdl{}2 \PYZcb{}}\PY{l+s}{\PYZsq{}}\PY{p}{)}
             \PY{n}{sds\PYZus{}names}\PY{o}{=}\PY{l+s}{\PYZdq{}}\PY{l+s}{\PYZdl{}sds\PYZus{}names \PYZdl{}sds}\PY{l+s}{\PYZdq{}}
         \PY{n}{done}
         
         \PY{n}{echo} \PY{l+s}{\PYZdq{}}\PY{l+s}{All subdatasets:}\PY{l+s}{\PYZdq{}}
         \PY{n}{echo} \PY{err}{\PYZdl{}}\PY{n}{sds\PYZus{}names}
\end{Verbatim}

    \begin{Verbatim}[commandchars=\\\{\}]
SUBDATASET\_1\_NAME=HDF4\_EOS:EOS\_GRID:"lndsr.LE70200472014061EDC00.hdf":Grid:band1
  SUBDATASET\_2\_NAME=HDF4\_EOS:EOS\_GRID:"lndsr.LE70200472014061EDC00.hdf":Grid:band2
  SUBDATASET\_3\_NAME=HDF4\_EOS:EOS\_GRID:"lndsr.LE70200472014061EDC00.hdf":Grid:band3
  SUBDATASET\_4\_NAME=HDF4\_EOS:EOS\_GRID:"lndsr.LE70200472014061EDC00.hdf":Grid:band4
  SUBDATASET\_5\_NAME=HDF4\_EOS:EOS\_GRID:"lndsr.LE70200472014061EDC00.hdf":Grid:band5
  SUBDATASET\_6\_NAME=HDF4\_EOS:EOS\_GRID:"lndsr.LE70200472014061EDC00.hdf":Grid:band7
  SUBDATASET\_15\_NAME=HDF4\_EOS:EOS\_GRID:"lndsr.LE70200472014061EDC00.hdf":Grid:band6
  SUBDATASET\_17\_NAME=HDF4\_EOS:EOS\_GRID:"lndsr.LE70200472014061EDC00.hdf":Grid:fmask\_band
All subdatasets:
HDF4\_EOS:EOS\_GRID:"lndsr.LE70200472014061EDC00.hdf":Grid:band1 HDF4\_EOS:EOS\_GRID:"lndsr.LE70200472014061EDC00.hdf":Grid:band2 HDF4\_EOS:EOS\_GRID:"lndsr.LE70200472014061EDC00.hdf":Grid:band3 HDF4\_EOS:EOS\_GRID:"lndsr.LE70200472014061EDC00.hdf":Grid:band4 HDF4\_EOS:EOS\_GRID:"lndsr.LE70200472014061EDC00.hdf":Grid:band5 HDF4\_EOS:EOS\_GRID:"lndsr.LE70200472014061EDC00.hdf":Grid:band7 HDF4\_EOS:EOS\_GRID:"lndsr.LE70200472014061EDC00.hdf":Grid:band6 HDF4\_EOS:EOS\_GRID:"lndsr.LE70200472014061EDC00.hdf":Grid:fmask\_band
    \end{Verbatim}

    gdal\_merge.py

Now that we know which subdatasets we would like to include, we can
create our ``layer stack'' using \textbf{\emph{gdal\_merge.py}} (see
http://www.gdal.org/gdal\_merge.html)

    \begin{Verbatim}[commandchars=\\\{\}]
{\color{incolor}In [{\color{incolor}32}]:} \PY{o}{\PYZpc{}\PYZpc{}}\PY{k}{bash}
         \PY{n}{gdal\PYZus{}merge}\PY{o}{.}\PY{n}{py} \PY{o}{\PYZhy{}}\PY{o}{\PYZhy{}}\PY{n}{help}
\end{Verbatim}

    \begin{Verbatim}[commandchars=\\\{\}]
Unrecognised command option: --help
Usage: gdal\_merge.py [-o out\_filename] [-of out\_format] [-co NAME=VALUE]*
                     [-ps pixelsize\_x pixelsize\_y] [-tap] [-separate] [-q] [-v] [-pct]
                     [-ul\_lr ulx uly lrx lry] [-init "value [value\ldots]"]
                     [-n nodata\_value] [-a\_nodata output\_nodata\_value]
                     [-ot datatype] [-createonly] input\_files
                     [--help-general]
    \end{Verbatim}

    To create an ``ENVI'' formatted binary image, with band interleave by
pixel interleave, we can modify our previous code:

    \begin{Verbatim}[commandchars=\\\{\}]
{\color{incolor}In [{\color{incolor}49}]:} \PY{o}{\PYZpc{}\PYZpc{}}\PY{k}{bash}
         \PY{n}{cd} \PY{o}{/}\PY{n}{projectnb}\PY{o}{/}\PY{n}{landsat}\PY{o}{/}\PY{n}{users}\PY{o}{/}\PY{n}{ceholden}\PY{o}{/}\PY{l+m+mi}{2014}\PY{n}{\PYZus{}Landsat\PYZus{}Preprocess}\PY{o}{/}\PY{l+m+mi}{5}\PY{n}{\PYZus{}Stack}\PY{o}{/}\PY{n}{LE70200472014061EDC00}
         
         \PY{n}{image}\PY{o}{=}\PY{n}{lndsr}\PY{o}{.}\PY{n}{LE70200472014061EDC00}\PY{o}{.}\PY{n}{hdf}
         \PY{n}{output}\PY{o}{=}\PY{n}{LE70200472014061EDC00\PYZus{}test\PYZus{}stack}\PY{o}{.}\PY{n}{bip}
         
         \PY{n}{bands}\PY{o}{=}\PY{l+s}{\PYZdq{}}\PY{l+s}{1 2 3 4 5 6 15 17}\PY{l+s}{\PYZdq{}}
         
         \PY{n}{sds\PYZus{}names}\PY{o}{=}\PY{l+s}{\PYZdq{}}\PY{l+s}{\PYZdq{}}
         
         \PY{k}{for} \PY{n}{b} \PY{o+ow}{in} \PY{err}{\PYZdl{}}\PY{n}{bands}\PY{p}{;} \PY{n}{do}
             \PY{n}{sds}\PY{o}{=}\PY{err}{\PYZdl{}}\PY{p}{(}\PY{n}{gdalinfo} \PY{err}{\PYZdl{}}\PY{n}{image} \PY{o}{|} \PY{n}{grep} \PY{l+s}{\PYZdq{}}\PY{l+s}{SUBDATASET\PYZus{}\PYZdl{}\PYZob{}b\PYZcb{}\PYZus{}NAME}\PY{l+s}{\PYZdq{}} \PY{o}{|} \PY{n}{awk} \PY{o}{\PYZhy{}}\PY{n}{F} \PY{l+s}{\PYZsq{}}\PY{l+s}{=}\PY{l+s}{\PYZsq{}} \PY{l+s}{\PYZsq{}}\PY{l+s}{\PYZob{} print \PYZdl{}2 \PYZcb{}}\PY{l+s}{\PYZsq{}}\PY{p}{)}
             \PY{n}{sds\PYZus{}names}\PY{o}{=}\PY{l+s}{\PYZdq{}}\PY{l+s}{\PYZdl{}sds\PYZus{}names \PYZdl{}sds}\PY{l+s}{\PYZdq{}}
         \PY{n}{done}
         
         \PY{c}{\PYZsh{} Create the stack}
         \PY{n}{gdal\PYZus{}merge}\PY{o}{.}\PY{n}{py} \PY{o}{\PYZhy{}}\PY{n}{o} \PY{n}{temp}\PY{o}{.}\PY{n}{gtif} \PY{o}{\PYZhy{}}\PY{n}{separate} \PY{o}{\PYZhy{}}\PY{n}{a\PYZus{}nodata} \PY{o}{\PYZhy{}}\PY{l+m+mi}{9999} \PY{err}{\PYZdl{}}\PY{n}{sds\PYZus{}names}
         
         \PY{c}{\PYZsh{} Correct the projection using gdal\PYZus{}translate}
         \PY{n}{gdal\PYZus{}translate} \PY{o}{\PYZhy{}}\PY{n}{of} \PY{n}{ENVI} \PY{o}{\PYZhy{}}\PY{n}{co} \PY{l+s}{\PYZdq{}}\PY{l+s}{INTERLEAVE=BIP}\PY{l+s}{\PYZdq{}} \PY{o}{\PYZhy{}}\PY{n}{a\PYZus{}srs} \PY{l+s}{\PYZsq{}}\PY{l+s}{+proj=utm +zone=16 +datum=WGS84 +units=m +no\PYZus{}defs }\PY{l+s}{\PYZsq{}} \PY{n}{temp}\PY{o}{.}\PY{n}{gtif} \PY{err}{\PYZdl{}}\PY{n}{output}
         
         \PY{c}{\PYZsh{} Delete the temporary GeoTIFF}
         \PY{n}{rm} \PY{n}{temp}\PY{o}{.}\PY{o}{*}
         
         \PY{c}{\PYZsh{} Query the new stack for metadata}
         \PY{n}{gdalinfo} \PY{err}{\PYZdl{}}\PY{n}{output}
\end{Verbatim}

    \begin{Verbatim}[commandchars=\\\{\}]
0\ldots10\ldots20\ldots30\ldots40\ldots50\ldots60\ldots70\ldots80\ldots90\ldots100 - done.
Input file size is 8121, 7101
0\ldots10\ldots20\ldots30\ldots40\ldots50\ldots60\ldots70\ldots80\ldots90\ldots100 - done.
Driver: ENVI/ENVI .hdr Labelled
Files: LE70200472014061EDC00\_test\_stack.bip
       LE70200472014061EDC00\_test\_stack.bip.aux.xml
       LE70200472014061EDC00\_test\_stack.hdr
Size is 8121, 7101
Coordinate System is:
PROJCS["WGS\_1984\_UTM\_Zone\_16N",
    GEOGCS["GCS\_WGS\_1984",
        DATUM["WGS\_1984",
            SPHEROID["WGS\_84",6378137,298.257223563]],
        PRIMEM["Greenwich",0],
        UNIT["Degree",0.017453292519943295]],
    PROJECTION["Transverse\_Mercator"],
    PARAMETER["latitude\_of\_origin",0],
    PARAMETER["central\_meridian",-87],
    PARAMETER["scale\_factor",0.9996],
    PARAMETER["false\_easting",500000],
    PARAMETER["false\_northing",0],
    UNIT["Meter",1]]
Origin = (78585.000000000000000,2186415.000000000000000)
Pixel Size = (30.000000000000000,-30.000000000000000)
Metadata:
  AREA\_OR\_POINT=Area
  Band\_1=Band 1
  Band\_2=Band 2
  Band\_3=Band 3
  Band\_4=Band 4
  Band\_5=Band 5
  Band\_6=Band 6
  Band\_7=Band 7
  Band\_8=Band 8
Image Structure Metadata:
  INTERLEAVE=PIXEL
Corner Coordinates:
Upper Left  (   78585.000, 2186415.000) ( 91d 1' 9.14"W, 19d43'42.02"N)
Lower Left  (   78585.000, 1973385.000) ( 90d58'25.95"W, 17d48'27.72"N)
Upper Right (  322215.000, 2186415.000) ( 88d41'48.80"W, 19d45'55.56"N)
Lower Right (  322215.000, 1973385.000) ( 88d40'39.72"W, 17d50'27.37"N)
Center      (  200400.000, 2079900.000) ( 89d50'31.44"W, 18d47'21.19"N)
Band 1 Block=8121x1 Type=Int16, ColorInterp=Undefined
  Description = Band 1
Band 2 Block=8121x1 Type=Int16, ColorInterp=Undefined
  Description = Band 2
Band 3 Block=8121x1 Type=Int16, ColorInterp=Undefined
  Description = Band 3
Band 4 Block=8121x1 Type=Int16, ColorInterp=Undefined
  Description = Band 4
Band 5 Block=8121x1 Type=Int16, ColorInterp=Undefined
  Description = Band 5
Band 6 Block=8121x1 Type=Int16, ColorInterp=Undefined
  Description = Band 6
Band 7 Block=8121x1 Type=Int16, ColorInterp=Undefined
  Description = Band 7
Band 8 Block=8121x1 Type=Int16, ColorInterp=Undefined
  Description = Band 8
    \end{Verbatim}

    Output extent

Since we will be utilizing a large number of images in whatever analysis
we perform, it would be convenient if all of the ``layer stack'' images
share a common image size and geographic extent. If they did not, then
our algorithm would have to perform some arithmetic to determine which
pixel it should read in from every image. Instead, it is much easier to
know that every row and column corresponds to the same extent on the
ground.

In order to do this, the simplest approach would be to define an output
extent based on one image. A more sophisticated approach would be to
define the output extent as the maximum extent required to fit all
images, but this is too difficult in Bash to demonstrate.

    \begin{Verbatim}[commandchars=\\\{\}]
{\color{incolor}In [{\color{incolor}53}]:} \PY{o}{\PYZpc{}\PYZpc{}}\PY{k}{bash}
         \PY{n}{cd} \PY{o}{/}\PY{n}{projectnb}\PY{o}{/}\PY{n}{landsat}\PY{o}{/}\PY{n}{users}\PY{o}{/}\PY{n}{ceholden}\PY{o}{/}\PY{l+m+mi}{2014}\PY{n}{\PYZus{}Landsat\PYZus{}Preprocess}\PY{o}{/}\PY{l+m+mi}{5}\PY{n}{\PYZus{}Stack}\PY{o}{/}
         
         \PY{c}{\PYZsh{} Define \PYZdq{}extent\PYZdq{} function (from https://github.com/dwtkns/gdal\PYZhy{}cheat\PYZhy{}sheet)}
         \PY{n}{function} \PY{n}{gdal\PYZus{}extent}\PY{p}{(}\PY{p}{)} \PY{p}{\PYZob{}}
             \PY{k}{if} \PY{p}{[} \PY{o}{\PYZhy{}}\PY{n}{z} \PY{l+s}{\PYZdq{}}\PY{l+s}{\PYZdl{}1}\PY{l+s}{\PYZdq{}} \PY{p}{]}\PY{p}{;} \PY{n}{then} 
                 \PY{n}{echo} \PY{l+s}{\PYZdq{}}\PY{l+s}{Missing arguments. Syntax:}\PY{l+s}{\PYZdq{}}
                 \PY{n}{echo} \PY{l+s}{\PYZdq{}}\PY{l+s}{  gdal\PYZus{}extent \PYZlt{}input\PYZus{}raster\PYZgt{}}\PY{l+s}{\PYZdq{}}
                 \PY{k}{return}
             \PY{n}{fi}
             \PY{n}{EXTENT}\PY{o}{=}\PY{err}{\PYZdl{}}\PY{p}{(}\PY{n}{gdalinfo} \PY{err}{\PYZdl{}}\PY{l+m+mi}{1} \PY{o}{|}\PYZbs{}
                 \PY{n}{grep} \PY{l+s}{\PYZdq{}}\PY{l+s}{Upper Left}\PY{l+s}{\PYZbs{}}\PY{l+s}{|Lower Right}\PY{l+s}{\PYZdq{}} \PY{o}{|}\PYZbs{}
                 \PY{n}{sed} \PY{l+s}{\PYZdq{}}\PY{l+s}{s/Upper Left  //g;s/Lower Right //g;s/).*//g}\PY{l+s}{\PYZdq{}} \PY{o}{|}\PYZbs{}
                 \PY{n}{tr} \PY{l+s}{\PYZdq{}}\PY{l+s+se}{\PYZbs{}n}\PY{l+s}{\PYZdq{}} \PY{l+s}{\PYZdq{}}\PY{l+s}{ }\PY{l+s}{\PYZdq{}} \PY{o}{|}\PYZbs{}
                 \PY{n}{sed} \PY{l+s}{\PYZsq{}}\PY{l+s}{s/ *\PYZdl{}//g}\PY{l+s}{\PYZsq{}} \PY{o}{|}\PYZbs{}
                 \PY{n}{tr} \PY{o}{\PYZhy{}}\PY{n}{d} \PY{l+s}{\PYZdq{}}\PY{l+s}{[(,]}\PY{l+s}{\PYZdq{}}\PY{p}{)}
             \PY{n}{echo} \PY{o}{\PYZhy{}}\PY{n}{n} \PY{l+s}{\PYZdq{}}\PY{l+s}{\PYZdl{}EXTENT}\PY{l+s}{\PYZdq{}}
         \PY{p}{\PYZcb{}}
         
         \PY{c}{\PYZsh{} Define \PYZdq{}subdataset\PYZdq{} function}
         \PY{n}{function} \PY{n}{get\PYZus{}sds}\PY{p}{(}\PY{p}{)} \PY{p}{\PYZob{}}
             \PY{k}{if} \PY{p}{[} \PY{o}{\PYZhy{}}\PY{n}{z} \PY{l+s}{\PYZdq{}}\PY{l+s}{\PYZdl{}1}\PY{l+s}{\PYZdq{}} \PY{p}{]} \PY{o}{|}\PY{o}{|} \PY{p}{[} \PY{o}{\PYZhy{}}\PY{n}{z} \PY{l+s}{\PYZdq{}}\PY{l+s}{\PYZdl{}2}\PY{l+s}{\PYZdq{}} \PY{p}{]}\PY{p}{;} \PY{n}{then}
                 \PY{n}{echo} \PY{l+s}{\PYZdq{}}\PY{l+s}{Missing arguments. Usage:}\PY{l+s}{\PYZdq{}}
                 \PY{n}{echo} \PY{l+s}{\PYZdq{}}\PY{l+s}{    get\PYZus{}sds \PYZlt{}raster\PYZgt{} \PYZlt{}bands\PYZgt{}}\PY{l+s}{\PYZdq{}}
                 \PY{k}{return}
             \PY{n}{fi}
         
             \PY{n}{sds\PYZus{}names}\PY{o}{=}\PY{l+s}{\PYZdq{}}\PY{l+s}{\PYZdq{}}
             \PY{k}{for} \PY{n}{b} \PY{o+ow}{in} \PY{err}{\PYZdl{}}\PY{l+m+mi}{2}\PY{p}{;} \PY{n}{do}
                 \PY{n}{sds}\PY{o}{=}\PY{err}{\PYZdl{}}\PY{p}{(}\PY{n}{gdalinfo} \PY{err}{\PYZdl{}}\PY{l+m+mi}{1} \PY{o}{|} \PY{n}{grep} \PY{l+s}{\PYZdq{}}\PY{l+s}{SUBDATASET\PYZus{}\PYZdl{}\PYZob{}b\PYZcb{}\PYZus{}NAME}\PY{l+s}{\PYZdq{}} \PY{o}{|} \PY{n}{awk} \PY{o}{\PYZhy{}}\PY{n}{F} \PY{l+s}{\PYZsq{}}\PY{l+s}{=}\PY{l+s}{\PYZsq{}} \PY{l+s}{\PYZsq{}}\PY{l+s}{\PYZob{} print \PYZdl{}2 \PYZcb{}}\PY{l+s}{\PYZsq{}}\PY{p}{)}
                 \PY{n}{sds\PYZus{}names}\PY{o}{=}\PY{l+s}{\PYZdq{}}\PY{l+s}{\PYZdl{}sds\PYZus{}names \PYZdl{}sds}\PY{l+s}{\PYZdq{}}
             \PY{n}{done}
             \PY{n}{echo} \PY{l+s}{\PYZdq{}}\PY{l+s}{\PYZdl{}sds\PYZus{}names}\PY{l+s}{\PYZdq{}}
         \PY{p}{\PYZcb{}}
         
         \PY{c}{\PYZsh{} Define index so we only grab extent from first}
         \PY{n}{i}\PY{o}{=}\PY{l+m+mi}{0}
         \PY{c}{\PYZsh{} Bands of interest}
         \PY{n}{bands}\PY{o}{=}\PY{l+s}{\PYZdq{}}\PY{l+s}{1 2 3 4 5 6 15 17}\PY{l+s}{\PYZdq{}}
         \PY{c}{\PYZsh{} Projection \PYZdq{}proj4\PYZdq{} string to use}
         \PY{n}{proj4}\PY{o}{=}\PY{l+s}{\PYZdq{}}\PY{l+s}{+proj=utm +zone=16 +datum=WGS84 +units=m +no\PYZus{}defs }\PY{l+s}{\PYZdq{}}
         
         \PY{c}{\PYZsh{} Loop through all image directories}
         \PY{k}{for} \PY{n}{img} \PY{o+ow}{in} \PY{err}{\PYZdl{}}\PY{p}{(}\PY{n}{find} \PY{o}{.}\PY{o}{/} \PY{o}{\PYZhy{}}\PY{n}{maxdepth} \PY{l+m+mi}{1} \PY{o}{\PYZhy{}}\PY{n}{name} \PY{l+s}{\PYZsq{}}\PY{l+s}{L*}\PY{l+s}{\PYZsq{}} \PY{o}{\PYZhy{}}\PY{n+nb}{type} \PY{n}{d}\PY{p}{)}\PY{p}{;} \PY{n}{do}
             \PY{n}{echo} \PY{l+s}{\PYZdq{}}\PY{l+s}{Creating layer stack for \PYZdl{}img}\PY{l+s}{\PYZdq{}}
             
             \PY{c}{\PYZsh{} Find LEDAPS file}
             \PY{n}{lndsr}\PY{o}{=}\PY{err}{\PYZdl{}}\PY{p}{(}\PY{n}{find} \PY{err}{\PYZdl{}}\PY{n}{img} \PY{o}{\PYZhy{}}\PY{n}{name} \PY{l+s}{\PYZsq{}}\PY{l+s}{lndsr*hdf}\PY{l+s}{\PYZsq{}}\PY{p}{)}
             \PY{c}{\PYZsh{} Get subdatasets \PYZhy{} quote bands so they\PYZsq{}re all sent as \PYZdl{}2}
             \PY{n}{subdatasets}\PY{o}{=}\PY{err}{\PYZdl{}}\PY{p}{(}\PY{n}{get\PYZus{}sds} \PY{err}{\PYZdl{}}\PY{n}{lndsr} \PY{l+s}{\PYZdq{}}\PY{l+s}{\PYZdl{}bands}\PY{l+s}{\PYZdq{}}\PY{p}{)}
         
             \PY{c}{\PYZsh{} Get extent of first image we find}
             \PY{k}{if} \PY{p}{[} \PY{err}{\PYZdl{}}\PY{n}{i} \PY{o}{\PYZhy{}}\PY{n}{eq} \PY{l+m+mi}{0} \PY{p}{]}\PY{p}{;} \PY{n}{then}
                 \PY{c}{\PYZsh{} Extent}
                 \PY{n}{sds}\PY{o}{=}\PY{err}{\PYZdl{}}\PY{p}{(}\PY{n}{echo} \PY{err}{\PYZdl{}}\PY{n}{subdatasets} \PY{o}{|} \PY{n}{awk} \PY{l+s}{\PYZsq{}}\PY{l+s}{\PYZob{} print \PYZdl{}1 \PYZcb{}}\PY{l+s}{\PYZsq{}}\PY{p}{)}
                 \PY{n}{extent}\PY{o}{=}\PY{err}{\PYZdl{}}\PY{p}{(}\PY{n}{gdal\PYZus{}extent} \PY{err}{\PYZdl{}}\PY{n}{sds}\PY{p}{)}
             \PY{n}{fi}
             
             \PY{c}{\PYZsh{} Find image Landsat ID for filename}
             \PY{n+nb}{id}\PY{o}{=}\PY{err}{\PYZdl{}}\PY{p}{(}\PY{n}{basename} \PY{err}{\PYZdl{}}\PY{n}{img}\PY{p}{)}
             
             \PY{n}{output}\PY{o}{=}\PY{err}{\PYZdl{}}\PY{n}{img}\PY{o}{/}\PY{err}{\PYZdl{}}\PY{p}{\PYZob{}}\PY{n+nb}{id}\PY{p}{\PYZcb{}}\PY{n}{\PYZus{}stack}
             
             \PY{c}{\PYZsh{} Create layer stack}
             \PY{n}{gdal\PYZus{}merge}\PY{o}{.}\PY{n}{py} \PY{o}{\PYZhy{}}\PY{n}{o} \PY{n}{temp}\PY{o}{.}\PY{n}{gtif} \PY{o}{\PYZhy{}}\PY{n}{ul\PYZus{}lr} \PY{err}{\PYZdl{}}\PY{n}{extent} \PY{o}{\PYZhy{}}\PY{n}{separate} \PY{o}{\PYZhy{}}\PY{n}{a\PYZus{}nodata} \PY{o}{\PYZhy{}}\PY{l+m+mi}{9999} \PY{err}{\PYZdl{}}\PY{n}{subdatasets}
             
             \PY{c}{\PYZsh{} Now translate this into ENVI BIP with correct projection}
             \PY{n}{gdal\PYZus{}translate} \PY{o}{\PYZhy{}}\PY{n}{of} \PY{n}{ENVI} \PY{o}{\PYZhy{}}\PY{n}{co} \PY{l+s}{\PYZdq{}}\PY{l+s}{INTERLEAVE=BIP}\PY{l+s}{\PYZdq{}} \PY{o}{\PYZhy{}}\PY{n}{a\PYZus{}srs} \PY{l+s}{\PYZdq{}}\PY{l+s}{\PYZdl{}proj4}\PY{l+s}{\PYZdq{}} \PY{n}{temp}\PY{o}{.}\PY{n}{gtif} \PY{err}{\PYZdl{}}\PY{n}{output}
         
             \PY{c}{\PYZsh{} Iterate i and delete temp file}
             \PY{n}{let} \PY{n}{i}\PY{o}{+}\PY{o}{=}\PY{l+m+mi}{1}
             \PY{n}{rm} \PY{n}{temp}\PY{o}{.}\PY{o}{*}
         \PY{n}{done}
\end{Verbatim}

    \begin{Verbatim}[commandchars=\\\{\}]
Creating layer stack for ./LE70200472014061EDC00
0\ldots10\ldots20\ldots30\ldots40\ldots50\ldots60\ldots70\ldots80\ldots90\ldots100 - done.
Input file size is 8121, 7101
0\ldots10\ldots20\ldots30\ldots40\ldots50\ldots60\ldots70\ldots80\ldots90\ldots100 - done.
Creating layer stack for ./LE70200472014077ASN00
0\ldots10\ldots20\ldots30\ldots40\ldots50\ldots60\ldots70\ldots80\ldots90\ldots100 - done.
Input file size is 8121, 7101
0\ldots10\ldots20\ldots30\ldots40\ldots50\ldots60\ldots70\ldots80\ldots90\ldots100 - done.
Creating layer stack for ./LE70200472014093ASN00
0\ldots10\ldots20\ldots30\ldots40\ldots50\ldots60\ldots70\ldots80\ldots90\ldots100 - done.
Input file size is 8121, 7101
0\ldots10\ldots20\ldots30\ldots40\ldots50\ldots60\ldots70\ldots80\ldots90\ldots100 - done.
Creating layer stack for ./LE70200472014109ASN00
0\ldots10\ldots20\ldots30\ldots40\ldots50\ldots60\ldots70\ldots80\ldots90\ldots100 - done.
Input file size is 8121, 7101
0\ldots10\ldots20\ldots30\ldots40\ldots50\ldots60\ldots70\ldots80\ldots90\ldots100 - done.
    \end{Verbatim}

    We can check out output to make sure they're all the same size and upper
left coordinate:

    \begin{Verbatim}[commandchars=\\\{\}]
{\color{incolor}In [{\color{incolor}55}]:} \PY{o}{\PYZpc{}\PYZpc{}}\PY{k}{bash}
         \PY{n}{cd} \PY{o}{/}\PY{n}{projectnb}\PY{o}{/}\PY{n}{landsat}\PY{o}{/}\PY{n}{users}\PY{o}{/}\PY{n}{ceholden}\PY{o}{/}\PY{l+m+mi}{2014}\PY{n}{\PYZus{}Landsat\PYZus{}Preprocess}\PY{o}{/}\PY{l+m+mi}{5}\PY{n}{\PYZus{}Stack}\PY{o}{/}
         
         \PY{k}{for} \PY{n}{stack} \PY{o+ow}{in} \PY{err}{\PYZdl{}}\PY{p}{(}\PY{n}{find} \PY{o}{.}\PY{o}{/} \PY{o}{\PYZhy{}}\PY{n}{name} \PY{l+s}{\PYZsq{}}\PY{l+s}{*stack}\PY{l+s}{\PYZsq{}}\PY{p}{)}\PY{p}{;} \PY{n}{do}
             \PY{n}{gdalinfo} \PY{err}{\PYZdl{}}\PY{n}{stack} \PY{o}{|} \PY{n}{grep} \PY{l+s}{\PYZdq{}}\PY{l+s}{Size is}\PY{l+s}{\PYZdq{}}
             \PY{n}{gdalinfo} \PY{err}{\PYZdl{}}\PY{n}{stack} \PY{o}{|} \PY{n}{grep} \PY{l+s}{\PYZdq{}}\PY{l+s}{Origin =}\PY{l+s}{\PYZdq{}}
             \PY{n}{du} \PY{o}{\PYZhy{}}\PY{n}{hs} \PY{err}{\PYZdl{}}\PY{n}{stack}
         \PY{n}{done}
\end{Verbatim}

    \begin{Verbatim}[commandchars=\\\{\}]
Size is 8121, 7101
Origin = (78585.000000000000000,2186415.000000000000000)
880M	./LE70200472014061EDC00/LE70200472014061EDC00\_stack
Size is 8121, 7101
Origin = (78585.000000000000000,2186415.000000000000000)
880M	./LE70200472014077ASN00/LE70200472014077ASN00\_stack
Size is 8121, 7101
Origin = (78585.000000000000000,2186415.000000000000000)
880M	./LE70200472014093ASN00/LE70200472014093ASN00\_stack
Size is 8121, 7101
Origin = (78585.000000000000000,2186415.000000000000000)
880M	./LE70200472014109ASN00/LE70200472014109ASN00\_stack
    \end{Verbatim}

    Complicated stuff!

Wow - that's a lot of Bash work! You could formalize these lines of code
into your own Bash script, of course, but it probably isn't the best
language for the job.

In order to create these layer stacks myself, I ended up writing a
heavily modified version of \textbf{\emph{gdal\_merge.py}} that does the
following: - Determines the output extent of your stack by: - the
maximum extent of all datasets - the minimum extent of all datasets - a
specified extent - a percentile of the extents (e.g., 1\% for minimum
and 99\% for maximum) - Treats HDF subdatasets as bands, removing
necessity for referencing them directly - Preserves band name metadata
(i.e., so your stack has the name ``band 1 reflectance'' from the LEDAPS
HDF file) - Applies NDV separately per band - Can overwrite UTM zone to
fix datum - Can ``resume'' if interrupted

If you'd like to use this solution, you may do so as follows:

    \begin{Verbatim}[commandchars=\\\{\}]
{\color{incolor}In [{\color{incolor}65}]:} \PY{o}{\PYZpc{}\PYZpc{}}\PY{k}{bash}
         \PY{n}{module} \PY{n}{load} \PY{n}{gdal}\PY{o}{/}\PY{l+m+mf}{1.10}\PY{o}{.}\PY{l+m+mi}{0}
         \PY{n}{module} \PY{n}{load} \PY{n}{batch\PYZus{}landsat}
         
         \PY{n}{landsat\PYZus{}stack}\PY{o}{.}\PY{n}{py} \PY{o}{\PYZhy{}}\PY{o}{\PYZhy{}}\PY{n}{help}
\end{Verbatim}

    \begin{Verbatim}[commandchars=\\\{\}]
Stack Landsat Data

Usage: landsat\_stack.py [options] (--max\_extent | --min\_extent | 
    --extent=<extent> | --percentile=<pct>) <location>

Options:
    -f --files=<files>\ldots       Files to stack [default: lndsr.*.hdf *Fmask]
    -b --bands=<bands>\ldots       Bands from files to stack [default: all]
    -d --dirs=<pattern>         Directory name pattern to search [default: L*]
    -o --output=<pattern>       Output filename pattern [default: *stack]
    -p --pickup                 Pickup / resume where left off 
    -n --ndv=<ndv>              No data value [default: 0]
    -u --utm=<zone>             Force a UTM zone (in WGS84)
    -e --exit-on-warn           Exit on warning messages
    --format=<format>           GDAL format [default: ENVI]
    --co=<creation options>     GDAL creation options [default: None]
    -v --verbose                Show verbose debugging messages
    -q --quiet                  Be quiet by not showing warnings
    --dry-run                   Dry run - don't actually stack
    -h --help                   Show help

Examples:
    landsat\_stack.py -vq -n "-9999; 255" -b "1 2 3 4 5 6 15; 1" --min\_extent ./
    \end{Verbatim}

    Example:

Let's say we want the same bands, but want to specify that the extent of
the images be the maximum of all images in the stack so nothing gets
clipped. We can also specify NoDataValues for each band:

    \begin{Verbatim}[commandchars=\\\{\}]
{\color{incolor}In [{\color{incolor}64}]:} \PY{o}{\PYZpc{}\PYZpc{}}\PY{k}{bash}
         \PY{n}{cd} \PY{o}{/}\PY{n}{projectnb}\PY{o}{/}\PY{n}{landsat}\PY{o}{/}\PY{n}{users}\PY{o}{/}\PY{n}{ceholden}\PY{o}{/}\PY{l+m+mi}{2014}\PY{n}{\PYZus{}Landsat\PYZus{}Preprocess}\PY{o}{/}\PY{l+m+mi}{5}\PY{n}{\PYZus{}Stack}\PY{o}{/}
         
         \PY{n}{module} \PY{n}{load} \PY{n}{gdal}\PY{o}{/}\PY{l+m+mf}{1.10}\PY{o}{.}\PY{l+m+mi}{0}
         \PY{n}{module} \PY{n}{load} \PY{n}{batch\PYZus{}landsat}
         
         \PY{n}{landsat\PYZus{}stack}\PY{o}{.}\PY{n}{py} \PY{o}{\PYZhy{}}\PY{n}{q} \PY{o}{\PYZhy{}}\PY{n}{p} \PY{o}{\PYZhy{}}\PY{o}{\PYZhy{}}\PY{n}{files} \PY{l+s}{\PYZdq{}}\PY{l+s}{lndsr*hdf; *Fmask}\PY{l+s}{\PYZdq{}} \PYZbs{}
             \PY{o}{\PYZhy{}}\PY{n}{b} \PY{l+s}{\PYZdq{}}\PY{l+s}{1 2 3 4 5 6 15; 1}\PY{l+s}{\PYZdq{}} \PYZbs{}
             \PY{o}{\PYZhy{}}\PY{n}{n} \PY{l+s}{\PYZdq{}}\PY{l+s}{\PYZhy{}9999 \PYZhy{}9999 \PYZhy{}9999 \PYZhy{}9999 \PYZhy{}9999 \PYZhy{}9999; 255}\PY{l+s}{\PYZdq{}} \PYZbs{}
             \PY{o}{\PYZhy{}}\PY{o}{\PYZhy{}}\PY{n}{utm} \PY{l+m+mi}{16} \PY{o}{\PYZhy{}}\PY{n}{o} \PY{l+s}{\PYZdq{}}\PY{l+s}{*\PYZus{}stack\PYZus{}chris}\PY{l+s}{\PYZdq{}} \PYZbs{}
             \PY{o}{\PYZhy{}}\PY{o}{\PYZhy{}}\PY{n}{format} \PY{l+s}{\PYZdq{}}\PY{l+s}{ENVI}\PY{l+s}{\PYZdq{}} \PY{o}{\PYZhy{}}\PY{o}{\PYZhy{}}\PY{n}{co} \PY{l+s}{\PYZdq{}}\PY{l+s}{INTERLEAVE=BIP}\PY{l+s}{\PYZdq{}} \PY{o}{\PYZhy{}}\PY{o}{\PYZhy{}}\PY{n}{max\PYZus{}extent} \PY{o}{.}\PY{o}{/}
\end{Verbatim}

    \begin{Verbatim}[commandchars=\\\{\}]
Found 4 Landsat images to stack.
Finding maximum extent
Landsat ID: LE70200472014061EDC00 updated maximum extent
Landsat ID: LE70200472014077ASN00 updated maximum extent
Landsat ID: LE70200472014093ASN00 updated maximum extent
Landsat ID: LE70200472014109ASN00 updated maximum extent

Stacking to extent:
	Upper Left: 77385.0,2187015.0
	Lower Right: 323115.0,1973385.0

Stacking images:
<--------------- 1 / 4 
Stacking:
 ./LE70200472014061EDC00/LE70200472014061EDC00\_stack\_chris

Target extent: [77385.0, 2187015.0, 323115.0, 1973385.0]
Output size: x=8191, y=7121
Output projection: 
 		PROJCS["UTM Zone 16, Northern Hemisphere",GEOGCS["WGS 84",DATUM["WGS\_1984",SPHEROID["WGS 84",6378137,298.257223563,AUTHORITY["EPSG","7030"]],TOWGS84[0,0,0,0,0,0,0],AUTHORITY["EPSG","6326"]],PRIMEM["Greenwich",0,AUTHORITY["EPSG","8901"]],UNIT["degree",0.0174532925199433,AUTHORITY["EPSG","9108"]],AUTHORITY["EPSG","4326"]],PROJECTION["Transverse\_Mercator"],PARAMETER["latitude\_of\_origin",0],PARAMETER["central\_meridian",-87],PARAMETER["scale\_factor",0.9996],PARAMETER["false\_easting",500000],PARAMETER["false\_northing",0],UNIT["Meter",1]]

<--------------- 2 / 4 
Stacking:
 ./LE70200472014077ASN00/LE70200472014077ASN00\_stack\_chris

Target extent: [77385.0, 2187015.0, 323115.0, 1973385.0]
Output size: x=8191, y=7121
Output projection: 
 		PROJCS["UTM Zone 16, Northern Hemisphere",GEOGCS["WGS 84",DATUM["WGS\_1984",SPHEROID["WGS 84",6378137,298.257223563,AUTHORITY["EPSG","7030"]],TOWGS84[0,0,0,0,0,0,0],AUTHORITY["EPSG","6326"]],PRIMEM["Greenwich",0,AUTHORITY["EPSG","8901"]],UNIT["degree",0.0174532925199433,AUTHORITY["EPSG","9108"]],AUTHORITY["EPSG","4326"]],PROJECTION["Transverse\_Mercator"],PARAMETER["latitude\_of\_origin",0],PARAMETER["central\_meridian",-87],PARAMETER["scale\_factor",0.9996],PARAMETER["false\_easting",500000],PARAMETER["false\_northing",0],UNIT["Meter",1]]

<--------------- 3 / 4 
Stacking:
 ./LE70200472014093ASN00/LE70200472014093ASN00\_stack\_chris

Target extent: [77385.0, 2187015.0, 323115.0, 1973385.0]
Output size: x=8191, y=7121
Output projection: 
 		PROJCS["UTM Zone 16, Northern Hemisphere",GEOGCS["WGS 84",DATUM["WGS\_1984",SPHEROID["WGS 84",6378137,298.257223563,AUTHORITY["EPSG","7030"]],TOWGS84[0,0,0,0,0,0,0],AUTHORITY["EPSG","6326"]],PRIMEM["Greenwich",0,AUTHORITY["EPSG","8901"]],UNIT["degree",0.0174532925199433,AUTHORITY["EPSG","9108"]],AUTHORITY["EPSG","4326"]],PROJECTION["Transverse\_Mercator"],PARAMETER["latitude\_of\_origin",0],PARAMETER["central\_meridian",-87],PARAMETER["scale\_factor",0.9996],PARAMETER["false\_easting",500000],PARAMETER["false\_northing",0],UNIT["Meter",1]]

<--------------- 4 / 4 
Stacking:
 ./LE70200472014109ASN00/LE70200472014109ASN00\_stack\_chris

Target extent: [77385.0, 2187015.0, 323115.0, 1973385.0]
Output size: x=8191, y=7121
Output projection: 
 		PROJCS["UTM Zone 16, Northern Hemisphere",GEOGCS["WGS 84",DATUM["WGS\_1984",SPHEROID["WGS 84",6378137,298.257223563,AUTHORITY["EPSG","7030"]],TOWGS84[0,0,0,0,0,0,0],AUTHORITY["EPSG","6326"]],PRIMEM["Greenwich",0,AUTHORITY["EPSG","8901"]],UNIT["degree",0.0174532925199433,AUTHORITY["EPSG","9108"]],AUTHORITY["EPSG","4326"]],PROJECTION["Transverse\_Mercator"],PARAMETER["latitude\_of\_origin",0],PARAMETER["central\_meridian",-87],PARAMETER["scale\_factor",0.9996],PARAMETER["false\_easting",500000],PARAMETER["false\_northing",0],UNIT["Meter",1]]



 --------------- REPORT --------------- 


Stacking completed successfully
    \end{Verbatim}

    Now let's check the ``landsat\_stack.py'' result:

    \begin{Verbatim}[commandchars=\\\{\}]
{\color{incolor}In [{\color{incolor}86}]:} \PY{o}{\PYZpc{}\PYZpc{}}\PY{k}{bash}
         \PY{n}{cd} \PY{o}{/}\PY{n}{projectnb}\PY{o}{/}\PY{n}{landsat}\PY{o}{/}\PY{n}{users}\PY{o}{/}\PY{n}{ceholden}\PY{o}{/}\PY{l+m+mi}{2014}\PY{n}{\PYZus{}Landsat\PYZus{}Preprocess}\PY{o}{/}\PY{l+m+mi}{5}\PY{n}{\PYZus{}Stack}\PY{o}{/}
         
         \PY{k}{for} \PY{n}{stack} \PY{o+ow}{in} \PY{err}{\PYZdl{}}\PY{p}{(}\PY{n}{find} \PY{o}{.}\PY{o}{/} \PY{o}{\PYZhy{}}\PY{n}{name} \PY{l+s}{\PYZsq{}}\PY{l+s}{*stack\PYZus{}chris}\PY{l+s}{\PYZsq{}}\PY{p}{)}\PY{p}{;} \PY{n}{do}
             \PY{n}{gdalinfo} \PY{err}{\PYZdl{}}\PY{n}{stack} \PY{o}{|} \PY{n}{grep} \PY{l+s}{\PYZdq{}}\PY{l+s}{Size is}\PY{l+s}{\PYZdq{}}
             \PY{n}{gdalinfo} \PY{err}{\PYZdl{}}\PY{n}{stack} \PY{o}{|} \PY{n}{grep} \PY{l+s}{\PYZdq{}}\PY{l+s}{Origin =}\PY{l+s}{\PYZdq{}}
             \PY{n}{du} \PY{o}{\PYZhy{}}\PY{n}{hs} \PY{err}{\PYZdl{}}\PY{n}{stack}
         \PY{n}{done}
\end{Verbatim}

    \begin{Verbatim}[commandchars=\\\{\}]
Size is 8191, 7121
Origin = (77385.000000000000000,2187015.000000000000000)
891M	./LE70200472014061EDC00/LE70200472014061EDC00\_stack\_chris
Size is 8191, 7121
Origin = (77385.000000000000000,2187015.000000000000000)
891M	./LE70200472014077ASN00/LE70200472014077ASN00\_stack\_chris
Size is 8191, 7121
Origin = (77385.000000000000000,2187015.000000000000000)
891M	./LE70200472014093ASN00/LE70200472014093ASN00\_stack\_chris
Size is 8191, 7121
Origin = (77385.000000000000000,2187015.000000000000000)
891M	./LE70200472014109ASN00/LE70200472014109ASN00\_stack\_chris
    \end{Verbatim}

    \subsection{Step 6. Run Fmask using custom parameters on images}

Depending on your scene location or tolerance for noise, you may wish to
run Fmask with a more conservative threshold.

Since this example is in the Yucatan Penninsula, my guess is that the
default Fmask cloud probability of 22.5 will miss a lot of small clouds.
Furthermore, we have a relatively large amount of data to work with, so
comission is not much of an issue.

To run Fmask on our cluster, you can run MATLAB as a single threaded
``batch style'' job using some command line arguments to MATLAB:

    \begin{Verbatim}[commandchars=\\\{\}]
{\color{incolor}In [{\color{incolor}70}]:} \PY{o}{\PYZpc{}\PYZpc{}}\PY{k}{bash}
         \PY{n}{module} \PY{n}{load} \PY{n}{gdal}\PY{o}{/}\PY{l+m+mf}{1.10}\PY{o}{.}\PY{l+m+mi}{0}
         
         \PY{n}{cd} \PY{o}{/}\PY{n}{projectnb}\PY{o}{/}\PY{n}{landsat}\PY{o}{/}\PY{n}{users}\PY{o}{/}\PY{n}{ceholden}\PY{o}{/}\PY{l+m+mi}{2014}\PY{n}{\PYZus{}Landsat\PYZus{}Preprocess}\PY{o}{/}\PY{l+m+mi}{6}\PY{n}{\PYZus{}Fmask}
         
         \PY{c}{\PYZsh{} Define MATLAB single threaded \PYZdq{}batch\PYZdq{} runtime}
         \PY{n}{ML}\PY{o}{=}\PY{l+s}{\PYZdq{}}\PY{l+s}{/usr/local/bin/matlab \PYZhy{}nodisplay \PYZhy{}nojvm \PYZhy{}singleCompThread \PYZhy{}r }\PY{l+s}{\PYZdq{}}
         
         \PY{c}{\PYZsh{} Define MATLAB command to run \PYZhy{} let\PYZsq{}s use 12.5 as cloud probabilty and dilate clouds and shadow by 5 and 4}
         \PY{n}{CMD}\PY{o}{=}\PY{l+s}{\PYZdq{}}\PY{l+s}{addpath(}\PY{l+s}{\PYZsq{}}\PY{l+s}{/usr3/graduate/zhuzhe/Algorithms/Fmask}\PY{l+s}{\PYZsq{}}\PY{l+s}{);clr\PYZus{}pct=autoFmask(5,4,3,12.5);fprintf(}\PY{l+s}{\PYZsq{}}\PY{l+s}{CLEAR=}\PY{l+s+si}{\PYZpc{}f}\PY{l+s+se}{\PYZbs{}n}\PY{l+s}{\PYZsq{}}\PY{l+s}{,clr\PYZus{}pct);exit;}\PY{l+s}{\PYZdq{}}
         
         \PY{c}{\PYZsh{} Loop over all images running Fmask}
         \PY{n}{here}\PY{o}{=}\PY{err}{\PYZdl{}}\PY{p}{(}\PY{n}{pwd}\PY{p}{)}
         
         \PY{k}{for} \PY{n}{img} \PY{o+ow}{in} \PY{err}{\PYZdl{}}\PY{p}{(}\PY{n}{find} \PY{o}{.}\PY{o}{/} \PY{o}{\PYZhy{}}\PY{n}{name} \PY{l+s}{\PYZsq{}}\PY{l+s}{L*}\PY{l+s}{\PYZsq{}} \PY{o}{\PYZhy{}}\PY{n+nb}{type} \PY{n}{d}\PY{p}{)}\PY{p}{;} \PY{n}{do}
             \PY{n}{cd} \PY{err}{\PYZdl{}}\PY{n}{img}
             
             \PY{c}{\PYZsh{} Let\PYZsq{}s not just run Fmask, but also capture clear percentage}
             \PY{n}{clear}\PY{o}{=}\PY{err}{\PYZdl{}}\PY{p}{(}\PY{err}{\PYZdl{}}\PY{n}{ML} \PY{err}{\PYZdl{}}\PY{n}{CMD} \PY{o}{|} \PY{n}{grep} \PY{l+s}{\PYZdq{}}\PY{l+s}{CLEAR=}\PY{l+s}{\PYZdq{}}\PY{p}{)}
             
             \PY{c}{\PYZsh{} Find Fmask}
             \PY{n}{fmask}\PY{o}{=}\PY{err}{\PYZdl{}}\PY{p}{(}\PY{n}{find} \PY{o}{.}\PY{o}{/} \PY{o}{\PYZhy{}}\PY{n}{name} \PY{l+s}{\PYZsq{}}\PY{l+s}{*Fmask}\PY{l+s}{\PYZsq{}}\PY{p}{)}
             \PY{c}{\PYZsh{} Add clear percent to metadata}
             \PY{n}{gdal\PYZus{}edit}\PY{o}{.}\PY{n}{py} \PY{o}{\PYZhy{}}\PY{n}{mo} \PY{l+s}{\PYZdq{}}\PY{l+s}{\PYZdl{}clear}\PY{l+s}{\PYZdq{}} \PY{err}{\PYZdl{}}\PY{n}{fmask}
             
             \PY{n}{cd} \PY{err}{\PYZdl{}}\PY{n}{here}
         \PY{n}{done}
         
         \PY{n}{echo} \PY{l+s}{\PYZdq{}}\PY{l+s}{Done!}\PY{l+s}{\PYZdq{}}
\end{Verbatim}

    \begin{Verbatim}[commandchars=\\\{\}]
Done!
    \end{Verbatim}

    \begin{Verbatim}[commandchars=\\\{\}]
{\color{incolor}In [{\color{incolor}71}]:} \PY{o}{\PYZpc{}\PYZpc{}}\PY{k}{bash}
         \PY{n}{cd} \PY{o}{/}\PY{n}{projectnb}\PY{o}{/}\PY{n}{landsat}\PY{o}{/}\PY{n}{users}\PY{o}{/}\PY{n}{ceholden}\PY{o}{/}\PY{l+m+mi}{2014}\PY{n}{\PYZus{}Landsat\PYZus{}Preprocess}\PY{o}{/}\PY{l+m+mi}{6}\PY{n}{\PYZus{}Fmask}
         
         \PY{k}{for} \PY{n}{fmask} \PY{o+ow}{in} \PY{err}{\PYZdl{}}\PY{p}{(}\PY{n}{find} \PY{o}{.}\PY{o}{/} \PY{o}{\PYZhy{}}\PY{n}{name} \PY{l+s}{\PYZsq{}}\PY{l+s}{*Fmask}\PY{l+s}{\PYZsq{}}\PY{p}{)}\PY{p}{;} \PY{n}{do}
             \PY{n}{echo} \PY{err}{\PYZdl{}}\PY{p}{(}\PY{n}{basename} \PY{err}{\PYZdl{}}\PY{n}{fmask}\PY{p}{)}
             \PY{n}{gdalinfo} \PY{err}{\PYZdl{}}\PY{n}{fmask} \PY{o}{|} \PY{n}{grep} \PY{l+s}{\PYZdq{}}\PY{l+s}{CLEAR}\PY{l+s}{\PYZdq{}}
         \PY{n}{done}
\end{Verbatim}

    \begin{Verbatim}[commandchars=\\\{\}]
LE70200472014061EDC00\_MTLFmask
  CLEAR=36.869076
LE70200472014077ASN00\_MTLFmask
  CLEAR=20.763808
LE70200472014093ASN00\_MTLFmask
  CLEAR=36.600765
LE70200472014109ASN00\_MTLFmask
  CLEAR=21.032762
    \end{Verbatim}

    Batch Fmask

Prior to the USGS providing surface reflectance results from LEDAPS, we
had to run LEDAPS ourselves here at BU.

To help facilitate the preprocessing of Landsat data, I created a script
that works on a directory full of Landsat image archives and
intelligently submits ``qsub'' jobs to preprocess each archive. Features
of the script include:

\begin{itemize}
\itemsep1pt\parskip0pt\parsep0pt
\item
  Stop preprocessing routine and create error log file if image is L1G
\item
  Delete TIF images after completed
\item
  Extract Landsat archives and organize by Path-Row/LandsatID
\item
  Perform LEDAPS and/or Fmask
\item
  Customize Fmask parameters
\item
  Set number of batch jobs running at one time on the cluster (useful
  since MATLAB runs out of licenses)
\end{itemize}

You can access this script by loading the module ``batch\_landsat'':

    \begin{Verbatim}[commandchars=\\\{\}]
{\color{incolor}In [{\color{incolor}88}]:} \PY{o}{\PYZpc{}\PYZpc{}}\PY{k}{bash}
         \PY{n}{module} \PY{n}{load} \PY{n}{batch\PYZus{}landsat}
         
         \PY{n}{landsatPrepSubmit}\PY{o}{.}\PY{n}{sh} \PY{o}{\PYZhy{}}\PY{n}{h}
\end{Verbatim}

    \begin{Verbatim}[commandchars=\\\{\}]
usage: /project/earth/packages/batch\_landsat/bin/landsatPrepSubmit.sh [options] image\_directory

    Author: Chris Holden (ceholden@bu.edu)

    Purpose:
    This script generates and manages the submissions of Landsat
    pre-processing (LEDAPS/Fmask) jobs to the Sun Grid Engine on BU's
    Katana cluster. 

    Note: requires script "landsatPrepQsub.sh" to also be in your path

    Options:
        -h  help
        -m  maximum SGE jobs at one time (default 2)
        -n  base for job names (default - landsat; cannot begin with a \#)
        -w  wait between qstat checks (default 60s)
        -d  delete TIF files?
        -c  cloud dilation parameter for FMASK (default 3)
        -s  shadow dilation parameter for FMASK (default 3)
        -p  cloud probability parameter for FMASK (default 22.5)
        -l  do LEDAPS? 1 - yes, 0 - no (default 1)
        -f  do FMask? 1 - yes, 0 - no (default 1)
        -x  do directory structure organization? (default 1)
            0 - no, just find tar.gz and move to their locations
            1 - yes, creates directory structure
        -g  check for L1G images and exit if found? (default 0)
        -e  send email? (default 1)
        -u  do unzipping? useful if data already extracted (default 1)
        -8  (BETA) use Qingsong's LC8 compatible version of LEDAPS (default 0)
        -2  Use Fmask 2.1 (default 0)

    Examples:
        Run LEDAPS and Fmask (3.2) using 10 jobs maximum and use custom Fmask 
        options on the folder "images":
        > landsatPrepSubmit.sh -m 10 -n myjob -w 60 -d -c 5 -s 4 -p 12.5 images/

        Run ONLY Fmask (3.2) on the files processed above, but this time with a 
        different Fmask cloud probability. Note the "-x" option:
        > landsatPrepSubmit.sh -m 10 -n myjob -c 5 -s 4 -p 22.5 images/P012-R031

        Run ONLY Fmask (3.2) on the files processed above, but with a different
        set of Fmask dilation values. Note that because we did not ask for the 
        TIF files to be deleted in the last run, they still exist and we will 
        not extract them ("-u 0").
        > landsatPrepSubmit.sh -m 10 -n myjob -c 3 -s 3 -p 22.5 images/P012-R031
    \end{Verbatim}

    While we could run LEDAPS with this tool, it is not longer required and
our version of LEDAPS is not as recently updated as the one running at
the USGS.

Instead, let's pretend we just downloaded new data and we'll unzip,
organize, and then run Fmask with custom parameters for each image:

\begin{quote}
\textgreater{} landsatPrepSubmit.sh -m 2 -d -c 5 -s 4 -p 12.5 -l 0
\end{quote}

This command will: - -m maximum of 2 qsub jobs - -d delete TIF files
when done - -c 5 dilate clouds by 5 pixels - -s 4 dilate shadows by 4
pixels - -p 12.5 cloud probability = 12.5 - -l 0 do not do LEDAPS

    \begin{Verbatim}[commandchars=\\\{\}]
{\color{incolor}In [{\color{incolor}89}]:} \PY{o}{\PYZpc{}\PYZpc{}}\PY{k}{bash}
         \PY{n}{cd} \PY{o}{/}\PY{n}{projectnb}\PY{o}{/}\PY{n}{landsat}\PY{o}{/}\PY{n}{users}\PY{o}{/}\PY{n}{ceholden}\PY{o}{/}\PY{l+m+mi}{2014}\PY{n}{\PYZus{}Landsat\PYZus{}Preprocess}\PY{o}{/}\PY{l+m+mi}{7}\PY{n}{\PYZus{}BatchPrep}
         
         \PY{n}{module} \PY{n}{load} \PY{n}{batch\PYZus{}landsat}
         
         \PY{n}{landsatPrepSubmit}\PY{o}{.}\PY{n}{sh} \PY{o}{\PYZhy{}}\PY{n}{m} \PY{l+m+mi}{2} \PY{o}{\PYZhy{}}\PY{n}{d} \PY{o}{\PYZhy{}}\PY{n}{c} \PY{l+m+mi}{5} \PY{o}{\PYZhy{}}\PY{n}{s} \PY{l+m+mi}{4} \PY{o}{\PYZhy{}}\PY{n}{p} \PY{l+m+mf}{12.5} \PY{o}{\PYZhy{}}\PY{n}{l} \PY{l+m+mi}{0} \PY{o}{\PYZhy{}}\PY{n}{e} \PY{l+m+mi}{0} \PY{o}{\PYZhy{}}\PY{n}{g} \PY{l+m+mi}{1} \PY{o}{.}\PY{o}{/}
\end{Verbatim}

    \begin{Verbatim}[commandchars=\\\{\}]
Scene directory:  /projectnb/landsat/users/ceholden/2014\_Landsat\_Preprocess/7\_BatchPrep
Maximum jobs:  2
Job basename:  landsat
Time between qstat checks:  60
Cloud dilation:  5
Shadow dilation:  4
Cloud probability:  12.5
Remove unneccessary files?:  1
Do LEDAPS?:  0
Do Fmask?:  1
Check for L1G?:  1
Send email?:  0
Do unzip?:  1
LC8 LEDAPS:  0
Use FMask 2.1sav:  0
Found 4 Landsat archives
Wrote jobs out\ldots
Jobs running: 0
<------------------------->
Your job 5773156 ("landsat.LE70200472014077-SC20140507171413") has been submitted
Submitted job \#1
<------------------------->
Jobs submitted:  1
Jobs on SGE:  0
Jobs running:  0
Jobs waiting:  0
<------------------------->
Your job 5773157 ("landsat.LE70200472014093-SC20140507171414") has been submitted
Submitted job \#2
<------------------------->
Jobs submitted:  2
Jobs on SGE:  1
Jobs running:  0
Jobs waiting:  1
<------------------------->
Jobs submitted:  2
Jobs on SGE:  2
Jobs running:  0
Jobs waiting:  2
<------------------------->
Jobs submitted:  2
Jobs on SGE:  2
Jobs running:  2
Jobs waiting:  0
<------------------------->
Jobs submitted:  2
Jobs on SGE:  2
Jobs running:  2
Jobs waiting:  0
<------------------------->
Your job 5773183 ("landsat.LE70200472014061-SC20140507171414") has been submitted
Submitted job \#3
<------------------------->
Jobs submitted:  3
Jobs on SGE:  1
Jobs running:  1
Jobs waiting:  0
<------------------------->
Jobs submitted:  3
Jobs on SGE:  2
Jobs running:  1
Jobs waiting:  1
<------------------------->
Your job 5773191 ("landsat.LE70200472014109-SC20140507171414") has been submitted
Submitted job \#4
<------------------------->
Jobs submitted:  4
Jobs on SGE:  1
Jobs running:  1
Jobs waiting:  0
All jobs submitted to SGE. Exiting.
    \end{Verbatim}

    Now that the job is done, let's look at our results:

    \begin{Verbatim}[commandchars=\\\{\}]
{\color{incolor}In [{\color{incolor}94}]:} \PY{o}{\PYZpc{}\PYZpc{}}\PY{k}{bash}
         \PY{n}{cd} \PY{o}{/}\PY{n}{projectnb}\PY{o}{/}\PY{n}{landsat}\PY{o}{/}\PY{n}{users}\PY{o}{/}\PY{n}{ceholden}\PY{o}{/}\PY{l+m+mi}{2014}\PY{n}{\PYZus{}Landsat\PYZus{}Preprocess}\PY{o}{/}\PY{l+m+mi}{7}\PY{n}{\PYZus{}BatchPrep}
         
         \PY{c}{\PYZsh{} Remember, this script organizes Landsat images by path\PYZhy{}row (e.g., P020\PYZhy{}R047) AND Landsat ID}
         \PY{n}{ls} \PY{o}{\PYZhy{}}\PY{n}{l} \PY{n}{P020}\PY{o}{\PYZhy{}}\PY{n}{R047}\PY{o}{/}
         
         \PY{n}{n}\PY{o}{=}\PY{err}{\PYZdl{}}\PY{p}{(}\PY{n}{find} \PY{n}{P020}\PY{o}{\PYZhy{}}\PY{n}{R047} \PY{o}{\PYZhy{}}\PY{n}{name} \PY{l+s}{\PYZsq{}}\PY{l+s}{*Fmask}\PY{l+s}{\PYZsq{}} \PY{o}{|} \PY{n}{wc} \PY{o}{\PYZhy{}}\PY{n}{l}\PY{p}{)}
         
         \PY{n}{echo} \PY{l+s}{\PYZdq{}}\PY{l+s}{\PYZdq{}}
         \PY{n}{echo} \PY{l+s}{\PYZdq{}}\PY{l+s}{We have \PYZdl{}n Fmask images}\PY{l+s}{\PYZdq{}}
\end{Verbatim}

    \begin{Verbatim}[commandchars=\\\{\}]
total 128
drwxr-sr-x 3 ceholden landsat 32768 May  8 14:04 LE70200472014061-SC20140507171414
drwxr-sr-x 3 ceholden landsat 32768 May  8 14:01 LE70200472014077-SC20140507171413
drwxr-sr-x 3 ceholden landsat 32768 May  8 14:01 LE70200472014093-SC20140507171414
drwxr-sr-x 3 ceholden landsat 32768 May  8 14:06 LE70200472014109-SC20140507171414

We have 4 Fmask images
    \end{Verbatim}

    Our results are organized by the filename of the archives without the
extension (i.e., no tar.gz).

To rename our files, we can write a simple Bash loop:

    \begin{Verbatim}[commandchars=\\\{\}]
{\color{incolor}In [{\color{incolor}98}]:} \PY{o}{\PYZpc{}\PYZpc{}}\PY{k}{bash}
         \PY{n}{cd} \PY{o}{/}\PY{n}{projectnb}\PY{o}{/}\PY{n}{landsat}\PY{o}{/}\PY{n}{users}\PY{o}{/}\PY{n}{ceholden}\PY{o}{/}\PY{l+m+mi}{2014}\PY{n}{\PYZus{}Landsat\PYZus{}Preprocess}\PY{o}{/}\PY{l+m+mi}{7}\PY{n}{\PYZus{}BatchPrep}\PY{o}{/}\PY{n}{P020}\PY{o}{\PYZhy{}}\PY{n}{R047}
         
         \PY{k}{for} \PY{n}{d} \PY{o+ow}{in} \PY{err}{\PYZdl{}}\PY{p}{(}\PY{n}{find} \PY{o}{.}\PY{o}{/} \PY{o}{\PYZhy{}}\PY{n}{name} \PY{l+s}{\PYZsq{}}\PY{l+s}{L*\PYZhy{}S*}\PY{l+s}{\PYZsq{}} \PY{o}{\PYZhy{}}\PY{n+nb}{type} \PY{n}{d}\PY{p}{)}\PY{p}{;} \PY{n}{do} 
             \PY{n+nb}{id}\PY{o}{=}\PY{err}{\PYZdl{}}\PY{p}{(}\PY{n}{find} \PY{err}{\PYZdl{}}\PY{n}{d} \PY{o}{\PYZhy{}}\PY{n}{name} \PY{l+s}{\PYZsq{}}\PY{l+s}{*MTL.txt}\PY{l+s}{\PYZsq{}} \PY{o}{\PYZhy{}}\PY{k}{exec} \PY{n}{basename} \PY{p}{\PYZob{}}\PY{p}{\PYZcb{}} \PYZbs{}\PY{p}{;}\PY{p}{)}
             \PY{n+nb}{id}\PY{o}{=}\PY{err}{\PYZdl{}}\PY{p}{(}\PY{n}{echo} \PY{err}{\PYZdl{}}\PY{n+nb}{id} \PY{o}{|} \PY{n}{awk} \PY{o}{\PYZhy{}}\PY{n}{F} \PY{l+s}{\PYZsq{}}\PY{l+s}{\PYZus{}}\PY{l+s}{\PYZsq{}} \PY{l+s}{\PYZsq{}}\PY{l+s}{\PYZob{} print \PYZdl{}1 \PYZcb{}}\PY{l+s}{\PYZsq{}}\PY{p}{)}
             
             \PY{n}{mv} \PY{err}{\PYZdl{}}\PY{n}{d} \PY{err}{\PYZdl{}}\PY{n+nb}{id}
         \PY{n}{done}
         
         \PY{n}{echo} \PY{l+s}{\PYZdq{}}\PY{l+s}{Done!}\PY{l+s}{\PYZdq{}}
         
         \PY{n}{ls} \PY{o}{\PYZhy{}}\PY{n}{l} \PY{o}{.}\PY{o}{/}
\end{Verbatim}

    \begin{Verbatim}[commandchars=\\\{\}]
Done!
total 128
drwxr-sr-x 3 ceholden landsat 32768 May  8 14:04 LE70200472014061EDC00
drwxr-sr-x 3 ceholden landsat 32768 May  8 14:01 LE70200472014077ASN00
drwxr-sr-x 3 ceholden landsat 32768 May  8 14:01 LE70200472014093ASN00
drwxr-sr-x 3 ceholden landsat 32768 May  8 14:06 LE70200472014109ASN00
    \end{Verbatim}

    \subsection{Step 7. Create spatial subset using extent or polygon ROI}

If you want to create a subset of your stacks, you can easily accomplish
this using one or two of the GDAL command line utilities. Subsets are
often very useful when you are rapidly iterating on your analysis and
don't need to extend it to your entire study area, or if your study area
is smaller than an entire Landsat image. In addition to just subsetting
the size of your raster datasets, you can use GDAL to actually mask
certain areas from your analysis.

Simple subsetting using extent

If you already have a bounding box defined (i.e., an extent listing
upper left and lower right coordinates), you can use:

\begin{quote}
\textbf{\emph{gdal\_translate}}
\end{quote}

to very easily subset your data. You can either define your subset in
units of pixels (-srcwin) or in units of actual coordinates (-projwin).

    \begin{Verbatim}[commandchars=\\\{\}]
{\color{incolor}In [{\color{incolor}123}]:} \PY{o}{\PYZpc{}\PYZpc{}}\PY{k}{bash}
          \PY{n}{gdal\PYZus{}translate} \PY{o}{\PYZhy{}}\PY{o}{\PYZhy{}}\PY{n}{help}
\end{Verbatim}

    \begin{Verbatim}[commandchars=\\\{\}]
Usage: gdal\_translate [--help-general] [--long-usage]
       [-ot \{Byte/Int16/UInt16/UInt32/Int32/Float32/Float64/
             CInt16/CInt32/CFloat32/CFloat64\}] [-strict]
       [-of format] [-b band] [-mask band] [-expand \{gray|rgb|rgba\}]
       [-outsize xsize[\%] ysize[\%]]
       [-unscale] [-scale [src\_min src\_max [dst\_min dst\_max]]]
       [-srcwin xoff yoff xsize ysize] [-projwin ulx uly lrx lry] [-epo] [-eco]
       [-a\_srs srs\_def] [-a\_ullr ulx uly lrx lry] [-a\_nodata value]
       [-gcp pixel line easting northing [elevation]]*
       [-mo "META-TAG=VALUE"]* [-q] [-sds]
       [-co "NAME=VALUE"]* [-stats]
       src\_dataset dst\_dataset
    \end{Verbatim}

    For example, let's say I want to create a subset somewhere in the center
of my stacked images:

    \begin{Verbatim}[commandchars=\\\{\}]
{\color{incolor}In [{\color{incolor}119}]:} \PY{o}{\PYZpc{}\PYZpc{}}\PY{k}{bash}
          \PY{n}{cd} \PY{o}{/}\PY{n}{projectnb}\PY{o}{/}\PY{n}{landsat}\PY{o}{/}\PY{n}{users}\PY{o}{/}\PY{n}{ceholden}\PY{o}{/}\PY{l+m+mi}{2014}\PY{n}{\PYZus{}Landsat\PYZus{}Preprocess}\PY{o}{/}\PY{l+m+mi}{8}\PY{n}{\PYZus{}Subset}
          
          \PY{c}{\PYZsh{} gdalinfo will list the center coordinate of our raster}
          \PY{n}{gdalinfo} \PY{n}{LE70200472014061EDC00}\PY{o}{/}\PY{o}{*}\PY{n}{stack} \PY{o}{|} \PY{n}{grep} \PY{l+s}{\PYZdq{}}\PY{l+s}{Center}\PY{l+s}{\PYZdq{}}
\end{Verbatim}

    \begin{Verbatim}[commandchars=\\\{\}]
Center      (  200400.000, 2079900.000) ( 89d50'31.44"W, 18d47'21.19"N)
    \end{Verbatim}

    Note that 200400.000, 2079900.000 is the center of a pixel closest to
our image's center so, we'll need to shift a half pixel to avoid
clipping in the middle of a pixel.

Let's make a 500x500 pixel raster centered on our image center:

    \begin{Verbatim}[commandchars=\\\{\}]
{\color{incolor}In [{\color{incolor}133}]:} \PY{o}{\PYZpc{}\PYZpc{}}\PY{k}{bash}
          \PY{n}{cd} \PY{o}{/}\PY{n}{projectnb}\PY{o}{/}\PY{n}{landsat}\PY{o}{/}\PY{n}{users}\PY{o}{/}\PY{n}{ceholden}\PY{o}{/}\PY{l+m+mi}{2014}\PY{n}{\PYZus{}Landsat\PYZus{}Preprocess}\PY{o}{/}\PY{l+m+mi}{8}\PY{n}{\PYZus{}Subset}
          
          \PY{n}{gdalinfo} \PY{n}{LE70200472014061EDC00}\PY{o}{/}\PY{o}{*}\PY{n}{stack} \PY{o}{|} \PY{n}{grep} \PY{l+s}{\PYZdq{}}\PY{l+s}{Center}\PY{l+s}{\PYZdq{}}
          
          \PY{c}{\PYZsh{} Center pixel ULx: 200385 }
          \PY{c}{\PYZsh{} Center pixel ULy: 2079915}
          
          \PY{c}{\PYZsh{} So if we resize by 250 pixels in each direction,}
          
          \PY{n}{ulx}\PY{o}{=}\PY{err}{\PYZdl{}}\PY{p}{(}\PY{n}{expr} \PY{l+m+mi}{200385} \PY{o}{\PYZhy{}} \PY{l+m+mi}{30} \PYZbs{}\PY{o}{*} \PY{l+m+mi}{250}\PY{p}{)}
          \PY{n}{uly}\PY{o}{=}\PY{err}{\PYZdl{}}\PY{p}{(}\PY{n}{expr} \PY{l+m+mi}{2079915} \PY{o}{+} \PY{l+m+mi}{30} \PYZbs{}\PY{o}{*} \PY{l+m+mi}{250}\PY{p}{)}
          \PY{n}{lrx}\PY{o}{=}\PY{err}{\PYZdl{}}\PY{p}{(}\PY{n}{expr} \PY{l+m+mi}{200385} \PY{o}{+} \PY{l+m+mi}{30} \PYZbs{}\PY{o}{*} \PY{l+m+mi}{250}\PY{p}{)}
          \PY{n}{lry}\PY{o}{=}\PY{err}{\PYZdl{}}\PY{p}{(}\PY{n}{expr} \PY{l+m+mi}{2079915} \PY{o}{\PYZhy{}} \PY{l+m+mi}{30} \PYZbs{}\PY{o}{*} \PY{l+m+mi}{250}\PY{p}{)}
          
          \PY{n}{echo} \PY{l+s}{\PYZdq{}}\PY{l+s}{\PYZdq{}}
          \PY{n}{echo} \PY{l+s}{\PYZdq{}}\PY{l+s}{Our new extent:}\PY{l+s}{\PYZdq{}}
          \PY{n}{extent}\PY{o}{=}\PY{l+s}{\PYZdq{}}\PY{l+s}{\PYZdl{}ulx \PYZdl{}uly \PYZdl{}lrx \PYZdl{}lry}\PY{l+s}{\PYZdq{}}
          \PY{n}{echo} \PY{err}{\PYZdl{}}\PY{n}{extent}
          
          \PY{c}{\PYZsh{} Perform subsetting on all of our stacked images}
          \PY{k}{for} \PY{n}{stack} \PY{o+ow}{in} \PY{err}{\PYZdl{}}\PY{p}{(}\PY{n}{find} \PY{o}{.}\PY{o}{/} \PY{o}{\PYZhy{}}\PY{n}{name} \PY{l+s}{\PYZsq{}}\PY{l+s}{*stack}\PY{l+s}{\PYZsq{}}\PY{p}{)}\PY{p}{;} \PY{n}{do}
              \PY{n+nb}{id}\PY{o}{=}\PY{err}{\PYZdl{}}\PY{p}{(}\PY{n}{basename} \PY{err}{\PYZdl{}}\PY{p}{(}\PY{n}{dirname} \PY{err}{\PYZdl{}}\PY{n}{stack}\PY{p}{)}\PY{p}{)}
          
              \PY{n}{echo} \PY{l+s}{\PYZdq{}}\PY{l+s}{Subsetting stack image for: \PYZdl{}id}\PY{l+s}{\PYZdq{}}
              
              \PY{n}{output}\PY{o}{=}\PY{err}{\PYZdl{}}\PY{p}{\PYZob{}}\PY{n}{stack}\PY{p}{\PYZcb{}}\PY{n}{\PYZus{}subset}\PY{o}{.}\PY{n}{gtif}
              
              \PY{n}{gdal\PYZus{}translate} \PY{o}{\PYZhy{}}\PY{n}{of} \PY{n}{GTiff} \PY{o}{\PYZhy{}}\PY{n}{projwin} \PY{err}{\PYZdl{}}\PY{n}{extent} \PY{err}{\PYZdl{}}\PY{n}{stack} \PY{err}{\PYZdl{}}\PY{n}{output}
              
              \PY{n}{echo} \PY{l+s}{\PYZdq{}}\PY{l+s}{\PYZdq{}}
          \PY{n}{done}
\end{Verbatim}

    \begin{Verbatim}[commandchars=\\\{\}]
Center      (  200400.000, 2079900.000) ( 89d50'31.44"W, 18d47'21.19"N)

Our new extent:
192885 2087415 207885 2072415
Subsetting stack image for: LE70200472014061EDC00
Input file size is 8121, 7101
Computed -srcwin 3810 3300 500 500 from projected window.
0\ldots10\ldots20\ldots30\ldots40\ldots50\ldots60\ldots70\ldots80\ldots90\ldots100 - done.

Subsetting stack image for: LE70200472014077ASN00
Input file size is 8121, 7101
Computed -srcwin 3810 3300 500 500 from projected window.
0\ldots10\ldots20\ldots30\ldots40\ldots50\ldots60\ldots70\ldots80\ldots90\ldots100 - done.

Subsetting stack image for: LE70200472014093ASN00
Input file size is 8121, 7101
Computed -srcwin 3810 3300 500 500 from projected window.
0\ldots10\ldots20\ldots30\ldots40\ldots50\ldots60\ldots70\ldots80\ldots90\ldots100 - done.

Subsetting stack image for: LE70200472014109ASN00
Input file size is 8121, 7101
Computed -srcwin 3810 3300 500 500 from projected window.
0\ldots10\ldots20\ldots30\ldots40\ldots50\ldots60\ldots70\ldots80\ldots90\ldots100 - done.
    \end{Verbatim}

    As you can see from \textbf{\emph{gdal\_translate}}'s output, it took
our ``projwin'' in projected coordinates (meters in our case, but it
could just as well be used for geographic data in decimal degrees) and
calculated the corresponding offsets and pixel size.

If you already knew a range of rows and columns from your raster, you
could specify the window you want to extract using:

\begin{quote}
-srcwin xoff yoff xsize ysize
\end{quote}

\begin{itemize}
\itemsep1pt\parskip0pt\parsep0pt
\item
  xoff offset from 0 in columns
\item
  yoff offset from 0 in rows
\item
  xsize number of columns to include
\item
  ysize number of rows to include
\end{itemize}

    More complicated subsetting - polygon \& masking outside of polygon

Let's suppose instead that you have some area of interest within your
Landsat image that you'd like to analyze. In this scenario, the rest of
the Landsat image would be a waste of disk space.

You could calculate the spatial extent needed to contain the polygon of
interest and simply run \textbf{\emph{gdal\_translate}} with the upper
left and lower right X and Y coordinates as shown above.

However, you could further limit your analysis by not only subsetting
your raster images, but by masking out areas that you're not interested
in. In order to do this, we will need to use

\begin{quote}
\textbf{\emph{gdalwarp}}
\end{quote}

\textbf{\emph{gdalwarp}} is mostly used for going from one projection to
another, but it can also be used with a polygon vector file to crop
images or to mask images.

\textbf{NOTE}: \textbf{\emph{gdalwarp}}'s intended goal is to do
reprojections. If we just want to clip our rasters, we will not be doing
any reprojections. Unfortunately, even if we do not tell the program to
reproject the data, it will recompute the pixel sizes and extent of our
raster (see \textbf{\emph{gdalwarp}} bug ticket
https://trac.osgeo.org/gdal/ticket/3947 and
\textbf{\emph{gdal\_translate}} enhancement ticket
https://trac.osgeo.org/gdal/ticket/4875)

We can sidestep this potential issue by feeding \textbf{\emph{gdalwarp}}
the correct extent and pixel size.

First thing to do will be to use GDAL's sister library, OGR, to extract
one polygon from a shapefile containing ``early action'' sites in
Mexico. OGR is a very useful library and has many powerful command line
utilities for working with vector data (see
http://www.gdal.org/ogr/ogr\_sql.html for information about OGR's
SQL-like queries).

    \begin{Verbatim}[commandchars=\\\{\}]
{\color{incolor}In [{\color{incolor}146}]:} \PY{o}{\PYZpc{}\PYZpc{}}\PY{k}{bash}
          \PY{n}{module} \PY{n}{load} \PY{n}{gdal}\PY{o}{/}\PY{l+m+mf}{1.10}\PY{o}{.}\PY{l+m+mi}{0}
          
          \PY{n}{cd} \PY{o}{/}\PY{n}{projectnb}\PY{o}{/}\PY{n}{landsat}\PY{o}{/}\PY{n}{users}\PY{o}{/}\PY{n}{ceholden}\PY{o}{/}\PY{l+m+mi}{2014}\PY{n}{\PYZus{}Landsat\PYZus{}Preprocess}\PY{o}{/}\PY{l+m+mi}{8}\PY{n}{\PYZus{}Subset}
          
          \PY{n}{ogrinfo} \PY{o}{\PYZhy{}}\PY{n}{al} \PY{o}{\PYZhy{}}\PY{n}{geom}\PY{o}{=}\PY{n}{NO} \PY{o}{\PYZhy{}}\PY{n}{sql} \PY{l+s}{\PYZsq{}}\PY{l+s}{SELECT * FROM mredd\PYZus{}aatr WHERE (ESTADO  LIKE }\PY{l+s}{\PYZdq{}}\PY{l+s}{\PYZpc{}}\PY{l+s}{Campeche}\PY{l+s}{\PYZdq{}}\PY{l+s}{)}\PY{l+s}{\PYZsq{}} \PY{n}{mredd\PYZus{}aatr}\PY{o}{.}\PY{n}{shp}
\end{Verbatim}

    \begin{Verbatim}[commandchars=\\\{\}]
INFO: Open of `mredd\_aatr.shp'
      using driver `ESRI Shapefile' successful.

Layer name: mredd\_aatr
Geometry: Polygon
Feature Count: 1
Extent: (-107.898611, 15.472142) - (-88.909351, 27.573289)
Layer SRS WKT:
GEOGCS["WGS 84",
    DATUM["WGS\_1984",
        SPHEROID["WGS 84",6378137,298.257223563,
            AUTHORITY["EPSG","7030"]],
        TOWGS84[0,0,0,0,0,0,0],
        AUTHORITY["EPSG","6326"]],
    PRIMEM["Greenwich",0,
        AUTHORITY["EPSG","8901"]],
    UNIT["degree",0.01745329251994328,
        AUTHORITY["EPSG","9122"]],
    AUTHORITY["EPSG","4326"]]
ID: Real (11.0)
SITIO: String (150.0)
POLIGONO: String (50.0)
ESTADO: String (50.0)
OGRFeature(mredd\_aatr):0
  ID (Real) = 1
  SITIO (String) = Sierra Pucc- Los Chenes
  POLIGONO (String) = -
  ESTADO (String) = Yucatán - Campeche
    \end{Verbatim}

    Now we can use \textbf{\emph{ogr2ogr}} to extract this one action site
into it's own shapefile. When we do this, we can also reproject our
shapefile into the UTM zone we've been using with our Landsat data:

    \begin{Verbatim}[commandchars=\\\{\}]
{\color{incolor}In [{\color{incolor}153}]:} \PY{o}{\PYZpc{}\PYZpc{}}\PY{k}{bash}
          \PY{n}{module} \PY{n}{load} \PY{n}{gdal}\PY{o}{/}\PY{l+m+mf}{1.10}\PY{o}{.}\PY{l+m+mi}{0}
          
          \PY{n}{cd} \PY{o}{/}\PY{n}{projectnb}\PY{o}{/}\PY{n}{landsat}\PY{o}{/}\PY{n}{users}\PY{o}{/}\PY{n}{ceholden}\PY{o}{/}\PY{l+m+mi}{2014}\PY{n}{\PYZus{}Landsat\PYZus{}Preprocess}\PY{o}{/}\PY{l+m+mi}{8}\PY{n}{\PYZus{}Subset}
          
          \PY{n}{ogr2ogr} \PY{o}{\PYZhy{}}\PY{n}{f} \PY{l+s}{\PYZdq{}}\PY{l+s}{ESRI Shapefile}\PY{l+s}{\PYZdq{}} \PY{o}{\PYZhy{}}\PY{n}{t\PYZus{}srs} \PY{n}{EPSG}\PY{p}{:}\PY{l+m+mi}{32616} \PYZbs{}
              \PY{o}{\PYZhy{}}\PY{n}{sql} \PY{l+s}{\PYZdq{}}\PY{l+s}{SELECT * FROM mredd\PYZus{}aatr WHERE (ESTADO  LIKE }\PY{l+s}{\PYZsq{}}\PY{l+s}{\PYZpc{}}\PY{l+s}{Campeche}\PY{l+s}{\PYZsq{}}\PY{l+s}{)}\PY{l+s}{\PYZdq{}} \PYZbs{}
              \PY{n}{mredd\PYZus{}yucatan}\PY{o}{.}\PY{n}{shp} \PY{n}{mredd\PYZus{}aatr}\PY{o}{.}\PY{n}{shp}
          
          \PY{n}{ogrinfo} \PY{o}{\PYZhy{}}\PY{n}{al} \PY{o}{\PYZhy{}}\PY{n}{geom}\PY{o}{=}\PY{n}{NO} \PY{n}{mredd\PYZus{}yucatan}\PY{o}{.}\PY{n}{shp}
\end{Verbatim}

    \begin{Verbatim}[commandchars=\\\{\}]
INFO: Open of `mredd\_yucatan.shp'
      using driver `ESRI Shapefile' successful.

Layer name: mredd\_yucatan
Geometry: Polygon
Feature Count: 1
Extent: (176449.259014, 2088249.065836) - (300170.064002, 2281342.856148)
Layer SRS WKT:
PROJCS["WGS\_1984\_UTM\_Zone\_16N",
    GEOGCS["GCS\_WGS\_1984",
        DATUM["WGS\_1984",
            SPHEROID["WGS\_84",6378137,298.257223563]],
        PRIMEM["Greenwich",0],
        UNIT["Degree",0.017453292519943295]],
    PROJECTION["Transverse\_Mercator"],
    PARAMETER["latitude\_of\_origin",0],
    PARAMETER["central\_meridian",-87],
    PARAMETER["scale\_factor",0.9996],
    PARAMETER["false\_easting",500000],
    PARAMETER["false\_northing",0],
    UNIT["Meter",1]]
ID: Real (11.0)
SITIO: String (150.0)
POLIGONO: String (50.0)
ESTADO: String (50.0)
OGRFeature(mredd\_yucatan):0
  ID (Real) = 1
  SITIO (String) = Sierra Pucc- Los Chenes
  POLIGONO (String) = -
  ESTADO (String) = Yucatán - Campeche
    \end{Verbatim}

    With our polygon shapefile ready to go, we can calculate an output
extent that clips to the polygon but preserves the spacing of the
Landsat data. Once again, we'll be using a simple Bash function from
Derek Watkin's GDAL/OGR cheat sheet
(https://github.com/dwtkns/gdal-cheat-sheet) to easily parse
\textbf{\emph{gdalinfo}} and \textbf{\emph{ogrinfo}}'s outputs into
extent strings:

    \begin{Verbatim}[commandchars=\\\{\}]
{\color{incolor}In [{\color{incolor}211}]:} \PY{o}{\PYZpc{}\PYZpc{}}\PY{k}{bash}
          \PY{n}{function} \PY{n}{gdal\PYZus{}extent}\PY{p}{(}\PY{p}{)} \PY{p}{\PYZob{}}
              \PY{k}{if} \PY{p}{[} \PY{o}{\PYZhy{}}\PY{n}{z} \PY{l+s}{\PYZdq{}}\PY{l+s}{\PYZdl{}1}\PY{l+s}{\PYZdq{}} \PY{p}{]}\PY{p}{;} \PY{n}{then} 
                  \PY{n}{echo} \PY{l+s}{\PYZdq{}}\PY{l+s}{Missing arguments. Syntax:}\PY{l+s}{\PYZdq{}}
                  \PY{n}{echo} \PY{l+s}{\PYZdq{}}\PY{l+s}{  gdal\PYZus{}extent \PYZlt{}input\PYZus{}raster\PYZgt{}}\PY{l+s}{\PYZdq{}}
                  \PY{k}{return}
              \PY{n}{fi}
              \PY{n}{EXTENT}\PY{o}{=}\PY{err}{\PYZdl{}}\PY{p}{(}\PY{n}{gdalinfo} \PY{err}{\PYZdl{}}\PY{l+m+mi}{1} \PY{o}{|}\PYZbs{}
                  \PY{n}{grep} \PY{l+s}{\PYZdq{}}\PY{l+s}{Upper Left}\PY{l+s}{\PYZbs{}}\PY{l+s}{|Lower Right}\PY{l+s}{\PYZdq{}} \PY{o}{|}\PYZbs{}
                  \PY{n}{sed} \PY{l+s}{\PYZdq{}}\PY{l+s}{s/Upper Left  //g;s/Lower Right //g;s/).*//g}\PY{l+s}{\PYZdq{}} \PY{o}{|}\PYZbs{}
                  \PY{n}{tr} \PY{l+s}{\PYZdq{}}\PY{l+s+se}{\PYZbs{}n}\PY{l+s}{\PYZdq{}} \PY{l+s}{\PYZdq{}}\PY{l+s}{ }\PY{l+s}{\PYZdq{}} \PY{o}{|}\PYZbs{}
                  \PY{n}{sed} \PY{l+s}{\PYZsq{}}\PY{l+s}{s/ *\PYZdl{}//g}\PY{l+s}{\PYZsq{}} \PY{o}{|}\PYZbs{}
                  \PY{n}{tr} \PY{o}{\PYZhy{}}\PY{n}{d} \PY{l+s}{\PYZdq{}}\PY{l+s}{[(,]}\PY{l+s}{\PYZdq{}}\PY{p}{)}
              \PY{n}{echo} \PY{o}{\PYZhy{}}\PY{n}{n} \PY{l+s}{\PYZdq{}}\PY{l+s}{\PYZdl{}EXTENT}\PY{l+s}{\PYZdq{}}
          \PY{p}{\PYZcb{}}
          
          \PY{n}{function} \PY{n}{ogr\PYZus{}extent}\PY{p}{(}\PY{p}{)} \PY{p}{\PYZob{}}
              \PY{k}{if} \PY{p}{[} \PY{o}{\PYZhy{}}\PY{n}{z} \PY{l+s}{\PYZdq{}}\PY{l+s}{\PYZdl{}1}\PY{l+s}{\PYZdq{}} \PY{p}{]}\PY{p}{;} \PY{n}{then} 
                  \PY{n}{echo} \PY{l+s}{\PYZdq{}}\PY{l+s}{Missing arguments. Syntax:}\PY{l+s}{\PYZdq{}}
                  \PY{n}{echo} \PY{l+s}{\PYZdq{}}\PY{l+s}{  ogr\PYZus{}extent \PYZlt{}input\PYZus{}vector\PYZgt{}}\PY{l+s}{\PYZdq{}}
                  \PY{k}{return}
              \PY{n}{fi}
              \PY{n}{EXTENT}\PY{o}{=}\PY{err}{\PYZdl{}}\PY{p}{(}\PY{n}{ogrinfo} \PY{o}{\PYZhy{}}\PY{n}{al} \PY{o}{\PYZhy{}}\PY{n}{so} \PY{err}{\PYZdl{}}\PY{l+m+mi}{1} \PY{o}{|}\PYZbs{}
                  \PY{n}{grep} \PY{n}{Extent} \PY{o}{|}\PYZbs{}
                  \PY{n}{sed} \PY{l+s}{\PYZsq{}}\PY{l+s}{s/Extent: //g}\PY{l+s}{\PYZsq{}} \PY{o}{|}\PYZbs{}
                  \PY{n}{sed} \PY{l+s}{\PYZsq{}}\PY{l+s}{s/(//g}\PY{l+s}{\PYZsq{}} \PY{o}{|}\PYZbs{}
                  \PY{n}{sed} \PY{l+s}{\PYZsq{}}\PY{l+s}{s/)//g}\PY{l+s}{\PYZsq{}} \PY{o}{|}\PYZbs{}
                  \PY{n}{sed} \PY{l+s}{\PYZsq{}}\PY{l+s}{s/ \PYZhy{} /, /g}\PY{l+s}{\PYZsq{}}\PY{p}{)}
              \PY{n}{EXTENT}\PY{o}{=}\PY{l+s+sb}{`echo \PYZdl{}EXTENT | awk \PYZhy{}F \PYZsq{},\PYZsq{} \PYZsq{}\PYZob{}print \PYZdl{}1 \PYZdq{} \PYZdq{} \PYZdl{}4 \PYZdq{} \PYZdq{} \PYZdl{}3 \PYZdq{} \PYZdq{} \PYZdl{}2\PYZcb{}\PYZsq{}`}
              \PY{n}{echo} \PY{o}{\PYZhy{}}\PY{n}{n} \PY{l+s}{\PYZdq{}}\PY{l+s}{\PYZdl{}EXTENT}\PY{l+s}{\PYZdq{}}
          \PY{p}{\PYZcb{}}
          
          \PY{n}{cd} \PY{o}{/}\PY{n}{projectnb}\PY{o}{/}\PY{n}{landsat}\PY{o}{/}\PY{n}{users}\PY{o}{/}\PY{n}{ceholden}\PY{o}{/}\PY{l+m+mi}{2014}\PY{n}{\PYZus{}Landsat\PYZus{}Preprocess}\PY{o}{/}\PY{l+m+mi}{8}\PY{n}{\PYZus{}Subset}
          
          \PY{n}{img\PYZus{}ext}\PY{o}{=}\PY{err}{\PYZdl{}}\PY{p}{(}\PY{n}{gdal\PYZus{}extent} \PY{n}{LE70200472014061EDC00}\PY{o}{/}\PY{n}{LE70200472014061EDC00\PYZus{}stack\PYZus{}chris}\PY{p}{)}
          \PY{n}{shp\PYZus{}ext}\PY{o}{=}\PY{err}{\PYZdl{}}\PY{p}{(}\PY{n}{ogr\PYZus{}extent} \PY{n}{mredd\PYZus{}yucatan}\PY{o}{.}\PY{n}{shp}\PY{p}{)}
          
          \PY{n}{pix}\PY{o}{=}\PY{err}{\PYZdl{}}\PY{p}{(}\PY{n}{gdalinfo} \PY{n}{LE70200472014061EDC00}\PY{o}{/}\PY{n}{LE70200472014061EDC00\PYZus{}stack\PYZus{}chris} \PYZbs{}
              \PY{o}{|} \PY{n}{grep} \PY{l+s}{\PYZdq{}}\PY{l+s}{Pixel Size}\PY{l+s}{\PYZdq{}} \PY{o}{|} \PY{n}{sed} \PY{l+s}{\PYZdq{}}\PY{l+s}{s/Pixel.*(//g;s/,/ /g;s/)//g}\PY{l+s}{\PYZdq{}}\PY{p}{)}
          \PY{n}{pix\PYZus{}sz}\PY{o}{=}\PY{l+s}{\PYZdq{}}\PY{l+s}{\PYZdl{}pix \PYZdl{}pix}\PY{l+s}{\PYZdq{}}
          
          \PY{n}{echo} \PY{l+s}{\PYZdq{}}\PY{l+s}{Extent of stacked images and extent of shapefile:}\PY{l+s}{\PYZdq{}}
          \PY{n}{echo} \PY{err}{\PYZdl{}}\PY{n}{img\PYZus{}ext}
          \PY{n}{echo} \PY{err}{\PYZdl{}}\PY{n}{shp\PYZus{}ext}
          
          \PY{n}{new\PYZus{}ext}\PY{o}{=}\PY{l+s}{\PYZdq{}}\PY{l+s}{\PYZdq{}}
          
          \PY{k}{for} \PY{n}{i} \PY{o+ow}{in} \PY{l+m+mi}{1} \PY{l+m+mi}{2} \PY{l+m+mi}{3} \PY{l+m+mi}{4}\PY{p}{;} \PY{n}{do}
              \PY{c}{\PYZsh{} Get the ith coordinate from sequence}
              \PY{n}{r}\PY{o}{=}\PY{err}{\PYZdl{}}\PY{p}{(}\PY{n}{echo} \PY{err}{\PYZdl{}}\PY{n}{img\PYZus{}ext} \PY{o}{|} \PY{n}{awk} \PY{o}{\PYZhy{}}\PY{n}{v} \PY{n}{i}\PY{o}{=}\PY{err}{\PYZdl{}}\PY{n}{i} \PY{l+s}{\PYZsq{}}\PY{l+s}{\PYZob{} print \PYZdl{}i \PYZcb{}}\PY{l+s}{\PYZsq{}}\PY{p}{)}
              \PY{n}{v}\PY{o}{=}\PY{err}{\PYZdl{}}\PY{p}{(}\PY{n}{echo} \PY{err}{\PYZdl{}}\PY{n}{shp\PYZus{}ext} \PY{o}{|} \PY{n}{awk} \PY{o}{\PYZhy{}}\PY{n}{v} \PY{n}{i}\PY{o}{=}\PY{err}{\PYZdl{}}\PY{n}{i} \PY{l+s}{\PYZsq{}}\PY{l+s}{\PYZob{} print \PYZdl{}i \PYZcb{}}\PY{l+s}{\PYZsq{}}\PY{p}{)}
              \PY{n}{pix}\PY{o}{=}\PY{err}{\PYZdl{}}\PY{p}{(}\PY{n}{echo} \PY{err}{\PYZdl{}}\PY{n}{pix\PYZus{}sz} \PY{o}{|} \PY{n}{awk} \PY{o}{\PYZhy{}}\PY{n}{v} \PY{n}{i}\PY{o}{=}\PY{err}{\PYZdl{}}\PY{n}{i} \PY{l+s}{\PYZsq{}}\PY{l+s}{\PYZob{} print \PYZdl{}i \PYZcb{}}\PY{l+s}{\PYZsq{}}\PY{p}{)}
              
              \PY{c}{\PYZsh{} Quick snippit of Python}
              \PY{n}{ext}\PY{o}{=}\PY{err}{\PYZdl{}}\PY{p}{(}\PY{n}{python} \PY{o}{\PYZhy{}}\PY{n}{c} \PY{l+s}{\PYZdq{}}\PY{l+s+se}{\PYZbs{}\PYZbs{}}
                  \PY{n}{offset}\PY{o}{=}\PY{n+nb}{int}\PY{p}{(}\PY{p}{(}\PY{err}{\PYZdl{}}\PY{n}{r} \PY{o}{\PYZhy{}} \PY{err}{\PYZdl{}}\PY{n}{v}\PY{p}{)} \PY{o}{/} \PY{err}{\PYZdl{}}\PY{n}{pix}\PY{p}{)}\PY{p}{;} \PYZbs{}
                  \PY{k}{print} \PY{err}{\PYZdl{}}\PY{n}{r} \PY{o}{\PYZhy{}} \PY{n}{offset} \PY{o}{*} \PY{err}{\PYZdl{}}\PY{n}{pix}\PYZbs{}
                  \PY{l+s}{\PYZdq{}}\PY{l+s}{)}
              \PY{n}{new\PYZus{}ext}\PY{o}{=}\PY{l+s}{\PYZdq{}}\PY{l+s}{\PYZdl{}new\PYZus{}ext \PYZdl{}ext}\PY{l+s}{\PYZdq{}}
          \PY{n}{done}
          
          \PY{n}{echo} \PY{l+s}{\PYZdq{}}\PY{l+s}{Calculated new extent:}\PY{l+s}{\PYZdq{}}
          \PY{n}{echo} \PY{err}{\PYZdl{}}\PY{n}{new\PYZus{}ext}
          
          \PY{c}{\PYZsh{} Now, unfortunately, gdalwarp wants us to specify xmin ymin xmax ymax}
          \PY{c}{\PYZsh{} In this case, this corresponds to the upper left X, lower right Y, lower right X, and upper left Y}
          \PY{n}{warp\PYZus{}ext}\PY{o}{=}\PY{err}{\PYZdl{}}\PY{p}{(}\PY{n}{echo} \PY{err}{\PYZdl{}}\PY{n}{new\PYZus{}ext} \PY{o}{|} \PY{n}{awk} \PY{l+s}{\PYZsq{}}\PY{l+s}{\PYZob{} print \PYZdl{}1 }\PY{l+s}{\PYZdq{}}\PY{l+s}{ }\PY{l+s}{\PYZdq{}}\PY{l+s}{ \PYZdl{}4 }\PY{l+s}{\PYZdq{}}\PY{l+s}{ }\PY{l+s}{\PYZdq{}}\PY{l+s}{ \PYZdl{}3 }\PY{l+s}{\PYZdq{}}\PY{l+s}{ }\PY{l+s}{\PYZdq{}}\PY{l+s}{ \PYZdl{}2 \PYZcb{}}\PY{l+s}{\PYZsq{}}\PY{p}{)}
          \PY{n}{echo} \PY{l+s}{\PYZdq{}}\PY{l+s}{gdalwarp extent:}\PY{l+s}{\PYZdq{}}
          \PY{n}{echo} \PY{err}{\PYZdl{}}\PY{n}{warp\PYZus{}ext}
          
          \PY{c}{\PYZsh{} Perform the clip:}
          
          \PY{k}{for} \PY{n}{stack} \PY{o+ow}{in} \PY{err}{\PYZdl{}}\PY{p}{(}\PY{n}{find} \PY{o}{.}\PY{o}{/} \PY{o}{\PYZhy{}}\PY{n}{name} \PY{l+s}{\PYZsq{}}\PY{l+s}{*stack}\PY{l+s}{\PYZsq{}}\PY{p}{)}\PY{p}{;} \PY{n}{do}
              \PY{n}{echo} \PY{l+s}{\PYZdq{}}\PY{l+s}{Clipping \PYZdl{}(basename \PYZdl{}(dirname \PYZdl{}stack))}\PY{l+s}{\PYZdq{}}
              
              \PY{n}{output}\PY{o}{=}\PY{err}{\PYZdl{}}\PY{p}{\PYZob{}}\PY{n}{stack}\PY{p}{\PYZcb{}}\PY{n}{\PYZus{}subset}
              
              \PY{n}{gdalwarp} \PY{o}{\PYZhy{}}\PY{n}{of} \PY{n}{ENVI} \PY{o}{\PYZhy{}}\PY{n}{co} \PY{l+s}{\PYZdq{}}\PY{l+s}{INTERLEAVE=BIP}\PY{l+s}{\PYZdq{}} \PY{o}{\PYZhy{}}\PY{n}{te} \PY{err}{\PYZdl{}}\PY{n}{warp\PYZus{}ext} \PY{o}{\PYZhy{}}\PY{n}{tr} \PY{l+m+mi}{30} \PY{l+m+mi}{30} \PYZbs{}
                  \PY{o}{\PYZhy{}}\PY{n}{cutline} \PY{n}{mredd\PYZus{}yucatan}\PY{o}{.}\PY{n}{shp} \PY{o}{\PYZhy{}}\PY{n}{cl} \PY{n}{mredd\PYZus{}yucatan} \PY{o}{\PYZhy{}}\PY{n}{crop\PYZus{}to\PYZus{}cutline} \PYZbs{}
                  \PY{o}{\PYZhy{}}\PY{n}{dstnodata} \PY{o}{\PYZhy{}}\PY{l+m+mi}{9999} \PY{o}{\PYZhy{}}\PY{n}{wm} \PY{l+m+mi}{2000} \PYZbs{}
                  \PY{err}{\PYZdl{}}\PY{n}{stack} \PY{err}{\PYZdl{}}\PY{n}{output}
          
          \PY{n}{done}
\end{Verbatim}

    \begin{Verbatim}[commandchars=\\\{\}]
Extent of stacked images and extent of shapefile:
77385.000 2187015.000 323115.000 1973385.000
176449.259014 2281342.856148 300170.064002 2088249.065836
Calculated new extent:
176445.0 2281335.0 300195.0 2088225.0
gdalwarp extent:
176445.0 2088225.0 300195.0 2281335.0
Clipping LE70200472014061EDC00
Creating output file that is 4124P x 6436L.
Processing input file ./LE70200472014061EDC00/LE70200472014061EDC00\_stack.
Using internal nodata values (eg. -9999) for image ./LE70200472014061EDC00/LE70200472014061EDC00\_stack.
0\ldots10\ldots20\ldots30\ldots40\ldots50\ldots60\ldots70\ldots80\ldots90\ldots100 - done.
Clipping LE70200472014077ASN00
Creating output file that is 4124P x 6436L.
Processing input file ./LE70200472014077ASN00/LE70200472014077ASN00\_stack.
Using internal nodata values (eg. -9999) for image ./LE70200472014077ASN00/LE70200472014077ASN00\_stack.
0\ldots10\ldots20\ldots30\ldots40\ldots50\ldots60\ldots70\ldots80\ldots90\ldots100 - done.
Clipping LE70200472014093ASN00
Creating output file that is 4124P x 6436L.
Processing input file ./LE70200472014093ASN00/LE70200472014093ASN00\_stack.
Using internal nodata values (eg. -9999) for image ./LE70200472014093ASN00/LE70200472014093ASN00\_stack.
0\ldots10\ldots20\ldots30\ldots40\ldots50\ldots60\ldots70\ldots80\ldots90\ldots100 - done.
Clipping LE70200472014109ASN00
Creating output file that is 4124P x 6436L.
Processing input file ./LE70200472014109ASN00/LE70200472014109ASN00\_stack.
Using internal nodata values (eg. -9999) for image ./LE70200472014109ASN00/LE70200472014109ASN00\_stack.
0\ldots10\ldots20\ldots30\ldots40\ldots50\ldots60\ldots70\ldots80\ldots90\ldots100 - done.
    \end{Verbatim}

    In practice, I don't actually calculate the correct extent that aligns
with my raster programatically - I just run the numbers myself. It's a
lot easier that way.

What did we get?

Before

Lots of clouds!

\subparagraph{After}

Well, at least it is smaller now

    \subsection{Step 8. Create ``preview'' images to help filter through
preprocessed imagery}

By now you are probably thinking - how am I to actually look at these
hundreds of images?

With the right data policies (i.e., free), hardware, and software tools,
the challenge of using Landsat data is no longer in getting the datasets
together; the challenge is looking at all of your source data and
results.

One easy way of assessing whether or not an individual image might be
useful or not is to create a ``browse'' or ``preview'' image of each
image in your time series. By ``browse'' or ``preview'', I mean a 3
color composite that is viewable using any number of image browsing
softwares.

GDAL tools

Creation of a ``browse'' image is possible by combining
\textbf{\emph{gdal\_merge.py}} with \textbf{\emph{gdal\_translate}}.
\textbf{\emph{gdal\_merge.py}} can be used to extract band combinations
from your stack images (e.g., my favorite 5, 4, and 3) into a new 3
color image. \textbf{\emph{gdal\_translate}} can then be used to resize
the image to a smaller resolution and scale the image numbers to 0-255.

I won't discuss this workflow here because it is very limited for time
series applications. For one, when you scale each image in
\textbf{\emph{gdal\_translate}}, the scaling is very simplistic and will
simply linearly scale the minimum and maximum of each band to fit
between 0 - 255. If we want to compare images through time, we would
need a constant stretch. Another limitation is that this workflow will
not help us identify how Fmask is performing.

If curious, a good description of this workflow is available here:

http://davidvhill.com/article/creating-browse-images-from-landsat-data

Custom tools

To help fulfill the limitations described above, I created my own
specific tool that has many possible customizations:

\begin{quote}
gen\_preview.py
\end{quote}

    \begin{Verbatim}[commandchars=\\\{\}]
{\color{incolor}In [{\color{incolor}213}]:} \PY{o}{\PYZpc{}\PYZpc{}}\PY{k}{bash}
          \PY{n}{module} \PY{n}{load} \PY{n}{gdal}\PY{o}{/}\PY{l+m+mf}{1.10}\PY{o}{.}\PY{l+m+mi}{0}
          \PY{n}{module} \PY{n}{load} \PY{n}{batch\PYZus{}landsat}
          
          \PY{n}{gen\PYZus{}preview}\PY{o}{.}\PY{n}{py} \PY{o}{\PYZhy{}}\PY{o}{\PYZhy{}}\PY{n}{help}
\end{Verbatim}

    \begin{Verbatim}[commandchars=\\\{\}]
Generate preview image

Usage:
    gen\_preview.py [options] (--linear\_pct <pct> | --histeq | --manual <minmax>) 
	<input> <output>

Options:
    -b --bands <bands>              Bands for output image [default: 3 2 1]
    --mask <mask>                   Mask band [default: 8]
    --maskval <value>\ldots            Mask band value [default: 2 3 4]
    --maskcol <r, g, b>             Mask color [default: 0, 0, 0]
    --ndv <value>                   No data value [default: 255]
    --threshold <percent>           Min unmasked data for output [default: 0]
    --srcwin <x y xsize ysize>      Window (in pixels) to subset
    --projwin <ulx uly lrx lry>     Window (in projected coordinates) to subset
    --resize\_pct <pct>              Image resize percent [default: 100]
    --resize\_method <method>        Image resize method [default: antialias]
    --format <format>               Output format [default: JPEG]
    -v --verbose                    Print (verbose) debugging messages
    -q --quiet                      Do not print except for warnings/errors
    -h --help                       Show help
    \end{Verbatim}

    With this program, we can not only create many ``browse'' images with a
consistent stretch, but we can:

\begin{itemize}
\itemsep1pt\parskip0pt\parsep0pt
\item
  --mask\\ specify Fmask band in our stack images
\item
  --maskval\\ specify which Fmask values we want to ignore
\item
  --threshold\\ specify minimum amount of unmasked pixels required to
  output preview
\item
  --resize\_pct\\ resize image by a percentage
\end{itemize}

For example:

    \begin{Verbatim}[commandchars=\\\{\}]
{\color{incolor}In [{\color{incolor}217}]:} \PY{o}{\PYZpc{}\PYZpc{}}\PY{k}{bash}
          \PY{n}{module} \PY{n}{load} \PY{n}{gdal}\PY{o}{/}\PY{l+m+mf}{1.10}\PY{o}{.}\PY{l+m+mi}{0}
          \PY{n}{module} \PY{n}{load} \PY{n}{batch\PYZus{}landsat}
          
          \PY{n}{cd} \PY{o}{/}\PY{n}{projectnb}\PY{o}{/}\PY{n}{landsat}\PY{o}{/}\PY{n}{users}\PY{o}{/}\PY{n}{ceholden}\PY{o}{/}\PY{l+m+mi}{2014}\PY{n}{\PYZus{}Landsat\PYZus{}Preprocess}\PY{o}{/}\PY{l+m+mi}{9}\PY{n}{\PYZus{}PreviewImage}
          
          \PY{k}{for} \PY{n}{stack} \PY{o+ow}{in} \PY{err}{\PYZdl{}}\PY{p}{(}\PY{n}{find} \PY{o}{.}\PY{o}{/} \PY{o}{\PYZhy{}}\PY{n}{name} \PY{l+s}{\PYZsq{}}\PY{l+s}{*stack}\PY{l+s}{\PYZsq{}}\PY{p}{)}\PY{p}{;} \PY{n}{do}
              \PY{n+nb}{id}\PY{o}{=}\PY{err}{\PYZdl{}}\PY{p}{(}\PY{n}{basename} \PY{err}{\PYZdl{}}\PY{p}{(}\PY{n}{dirname} \PY{err}{\PYZdl{}}\PY{n}{stack}\PY{p}{)}\PY{p}{)}
              \PY{n}{echo} \PY{l+s}{\PYZdq{}}\PY{l+s}{Making preview image for \PYZdl{}id}\PY{l+s}{\PYZdq{}}
              
              \PY{n}{gen\PYZus{}preview}\PY{o}{.}\PY{n}{py} \PY{o}{\PYZhy{}}\PY{o}{\PYZhy{}}\PY{n}{format} \PY{n}{PNG} \PY{o}{\PYZhy{}}\PY{n}{b} \PY{l+s}{\PYZdq{}}\PY{l+s}{5 4 3}\PY{l+s}{\PYZdq{}} \PY{o}{\PYZhy{}}\PY{o}{\PYZhy{}}\PY{n}{resize\PYZus{}pct} \PY{l+m+mi}{25} \PY{o}{\PYZhy{}}\PY{o}{\PYZhy{}}\PY{n}{manual} \PY{l+s}{\PYZdq{}}\PY{l+s}{0 8000}\PY{l+s}{\PYZdq{}} \PY{err}{\PYZdl{}}\PY{n}{stack} \PY{err}{\PYZdl{}}\PY{p}{\PYZob{}}\PY{n+nb}{id}\PY{p}{\PYZcb{}}\PY{n}{\PYZus{}preview}\PY{o}{.}\PY{n}{png}
              
          \PY{n}{done}
          
          \PY{n}{echo} \PY{l+s}{\PYZdq{}}\PY{l+s}{Done!}\PY{l+s}{\PYZdq{}}
\end{Verbatim}

    \begin{Verbatim}[commandchars=\\\{\}]
Making preview image for LE70200472014061EDC00
Making preview image for LE70200472014077ASN00
Making preview image for LE70200472014093ASN00
Making preview image for LE70200472014109ASN00
Done!
    \end{Verbatim}

    Results:

\paragraph{And if you really want to get creative\ldots{}}

    \begin{Verbatim}[commandchars=\\\{\}]
{\color{incolor}In [{\color{incolor}1}]:} \PY{k+kn}{import} \PY{n+nn}{io}
        \PY{k+kn}{import} \PY{n+nn}{base64}
        \PY{k+kn}{from} \PY{n+nn}{IPython.display} \PY{k+kn}{import} \PY{n}{HTML}
        
        \PY{n}{video} \PY{o}{=} \PY{n}{io}\PY{o}{.}\PY{n}{open}\PY{p}{(}\PY{l+s}{\PYZsq{}}\PY{l+s}{resources/movie.mp4}\PY{l+s}{\PYZsq{}}\PY{p}{,} \PY{l+s}{\PYZsq{}}\PY{l+s}{r+b}\PY{l+s}{\PYZsq{}}\PY{p}{)}\PY{o}{.}\PY{n}{read}\PY{p}{(}\PY{p}{)}
        \PY{n}{encoded} \PY{o}{=} \PY{n}{base64}\PY{o}{.}\PY{n}{b64encode}\PY{p}{(}\PY{n}{video}\PY{p}{)}
        \PY{n}{HTML}\PY{p}{(}\PY{n}{data}\PY{o}{=}\PY{l+s}{\PYZsq{}\PYZsq{}\PYZsq{}}\PY{l+s}{\PYZlt{}video alt=}\PY{l+s}{\PYZdq{}}\PY{l+s}{test}\PY{l+s}{\PYZdq{}}\PY{l+s}{ controls\PYZgt{}}
        \PY{l+s}{                \PYZlt{}source src=}\PY{l+s}{\PYZdq{}}\PY{l+s}{data:video/mp4;base64,\PYZob{}0\PYZcb{}}\PY{l+s}{\PYZdq{}}\PY{l+s}{ type=}\PY{l+s}{\PYZdq{}}\PY{l+s}{video/mp4}\PY{l+s}{\PYZdq{}}\PY{l+s}{ /\PYZgt{}}
        \PY{l+s}{             \PYZlt{}/video\PYZgt{}}\PY{l+s}{\PYZsq{}\PYZsq{}\PYZsq{}}\PY{o}{.}\PY{n}{format}\PY{p}{(}\PY{n}{encoded}\PY{o}{.}\PY{n}{decode}\PY{p}{(}\PY{l+s}{\PYZsq{}}\PY{l+s}{ascii}\PY{l+s}{\PYZsq{}}\PY{p}{)}\PY{p}{)}\PY{p}{)}
\end{Verbatim}

            \begin{Verbatim}[commandchars=\\\{\}]
{\color{outcolor}Out[{\color{outcolor}1}]:} <IPython.core.display.HTML at 0x20b2990>
\end{Verbatim}
        
    \subsection{Step 9. Summarize and document your dataset}

By now, you've probably seen how useful loops combined with grep, sed,
awk, etc. can be.

The following are some examples of how you might summarize your dataset
by querying the metadata provided with your Landsat images:

ACCA Cloud Cover

    \begin{Verbatim}[commandchars=\\\{\}]
{\color{incolor}In [{\color{incolor}228}]:} \PY{o}{\PYZpc{}\PYZpc{}}\PY{k}{bash}
          \PY{n}{cd} \PY{o}{/}\PY{n}{projectnb}\PY{o}{/}\PY{n}{landsat}\PY{o}{/}\PY{n}{users}\PY{o}{/}\PY{n}{ceholden}\PY{o}{/}\PY{l+m+mi}{2014}\PY{n}{\PYZus{}Landsat\PYZus{}Preprocess}\PY{o}{/}\PY{l+m+mi}{5}\PY{n}{\PYZus{}Stack}
          
          \PY{k}{for} \PY{n}{mtl} \PY{o+ow}{in} \PY{err}{\PYZdl{}}\PY{p}{(}\PY{n}{find} \PY{o}{.}\PY{o}{/} \PY{o}{\PYZhy{}}\PY{n}{name} \PY{l+s}{\PYZsq{}}\PY{l+s}{*MTL.txt}\PY{l+s}{\PYZsq{}}\PY{p}{)}\PY{p}{;} \PY{n}{do}
              \PY{n+nb}{id}\PY{o}{=}\PY{err}{\PYZdl{}}\PY{p}{(}\PY{n}{basename} \PY{err}{\PYZdl{}}\PY{p}{(}\PY{n}{dirname} \PY{err}{\PYZdl{}}\PY{n}{mtl}\PY{p}{)}\PY{p}{)}
              \PY{n}{echo} \PY{err}{\PYZdl{}}\PY{n+nb}{id} \PY{o}{\PYZhy{}} \PY{err}{\PYZdl{}}\PY{p}{(}\PY{n}{grep} \PY{l+s}{\PYZdq{}}\PY{l+s}{CLOUD\PYZus{}COVER}\PY{l+s}{\PYZdq{}} \PY{err}{\PYZdl{}}\PY{n}{mtl} \PY{o}{|} \PY{n}{tr} \PY{o}{\PYZhy{}}\PY{n}{d} \PY{l+s}{\PYZsq{}}\PY{l+s}{ }\PY{l+s}{\PYZsq{}} \PY{o}{|} \PY{n}{awk} \PY{o}{\PYZhy{}}\PY{n}{F} \PY{l+s}{\PYZsq{}}\PY{l+s}{=}\PY{l+s}{\PYZsq{}} \PY{l+s}{\PYZsq{}}\PY{l+s}{\PYZob{} print \PYZdl{}2 \PYZcb{}}\PY{l+s}{\PYZsq{}}\PY{p}{)}
          \PY{n}{done}
\end{Verbatim}

    \begin{Verbatim}[commandchars=\\\{\}]
LE70200472014061EDC00 - 21.00
LE70200472014077ASN00 - 59.00
LE70200472014093ASN00 - 17.00
LE70200472014109ASN00 - 40.00
    \end{Verbatim}

    Sun Elevation, Sun Azimuth

    \begin{Verbatim}[commandchars=\\\{\}]
{\color{incolor}In [{\color{incolor}227}]:} \PY{o}{\PYZpc{}\PYZpc{}}\PY{k}{bash}
          \PY{n}{cd} \PY{o}{/}\PY{n}{projectnb}\PY{o}{/}\PY{n}{landsat}\PY{o}{/}\PY{n}{users}\PY{o}{/}\PY{n}{ceholden}\PY{o}{/}\PY{l+m+mi}{2014}\PY{n}{\PYZus{}Landsat\PYZus{}Preprocess}\PY{o}{/}\PY{l+m+mi}{5}\PY{n}{\PYZus{}Stack}
          
          \PY{k}{for} \PY{n}{mtl} \PY{o+ow}{in} \PY{err}{\PYZdl{}}\PY{p}{(}\PY{n}{find} \PY{o}{.}\PY{o}{/} \PY{o}{\PYZhy{}}\PY{n}{name} \PY{l+s}{\PYZsq{}}\PY{l+s}{*MTL.txt}\PY{l+s}{\PYZsq{}}\PY{p}{)}\PY{p}{;} \PY{n}{do}
              \PY{n+nb}{id}\PY{o}{=}\PY{err}{\PYZdl{}}\PY{p}{(}\PY{n}{basename} \PY{err}{\PYZdl{}}\PY{p}{(}\PY{n}{dirname} \PY{err}{\PYZdl{}}\PY{n}{mtl}\PY{p}{)}\PY{p}{)}
              \PY{n}{az}\PY{o}{=}\PY{err}{\PYZdl{}}\PY{p}{(}\PY{n}{grep} \PY{l+s}{\PYZdq{}}\PY{l+s}{SUN\PYZus{}AZIMUTH}\PY{l+s}{\PYZdq{}} \PY{err}{\PYZdl{}}\PY{n}{mtl} \PY{o}{|} \PY{n}{tr} \PY{o}{\PYZhy{}}\PY{n}{d} \PY{l+s}{\PYZsq{}}\PY{l+s}{ }\PY{l+s}{\PYZsq{}} \PY{o}{|} \PY{n}{awk} \PY{o}{\PYZhy{}}\PY{n}{F} \PY{l+s}{\PYZsq{}}\PY{l+s}{=}\PY{l+s}{\PYZsq{}} \PY{l+s}{\PYZsq{}}\PY{l+s}{\PYZob{} print \PYZdl{}2 \PYZcb{}}\PY{l+s}{\PYZsq{}}\PY{p}{)}
              \PY{n}{el}\PY{o}{=}\PY{err}{\PYZdl{}}\PY{p}{(}\PY{n}{grep} \PY{l+s}{\PYZdq{}}\PY{l+s}{SUN\PYZus{}ELEVATION}\PY{l+s}{\PYZdq{}} \PY{err}{\PYZdl{}}\PY{n}{mtl} \PY{o}{|} \PY{n}{tr} \PY{o}{\PYZhy{}}\PY{n}{d} \PY{l+s}{\PYZsq{}}\PY{l+s}{ }\PY{l+s}{\PYZsq{}} \PY{o}{|} \PY{n}{awk} \PY{o}{\PYZhy{}}\PY{n}{F} \PY{l+s}{\PYZsq{}}\PY{l+s}{=}\PY{l+s}{\PYZsq{}} \PY{l+s}{\PYZsq{}}\PY{l+s}{\PYZob{} print \PYZdl{}2 \PYZcb{}}\PY{l+s}{\PYZsq{}}\PY{p}{)}
              
              \PY{n}{echo} \PY{l+s}{\PYZdq{}}\PY{l+s}{\PYZdl{}id, \PYZdl{}az, \PYZdl{}el}\PY{l+s}{\PYZdq{}}
          \PY{n}{done}
\end{Verbatim}

    \begin{Verbatim}[commandchars=\\\{\}]
LE70200472014061EDC00, 130.59617147, 52.22814557
LE70200472014077ASN00, 123.58852009, 57.11885778
LE70200472014093ASN00, 114.75958139, 61.65047059
LE70200472014109ASN00, 103.95380278, 65.12459343
    \end{Verbatim}

    Number of observations per pixel

If you want a map of how many observations there are per pixel, you
could loop through all images in your dataset and count the number of
times each pixel is not masked by Fmask.

In order to do this, I wrote the following program:

\begin{quote}
stack\_nobs.py
\end{quote}

    \begin{Verbatim}[commandchars=\\\{\}]
{\color{incolor}In [{\color{incolor}236}]:} \PY{o}{\PYZpc{}\PYZpc{}}\PY{k}{bash}
          \PY{n}{module} \PY{n}{load} \PY{n}{batch\PYZus{}landsat}
          
          \PY{n}{stack\PYZus{}nobs}\PY{o}{.}\PY{n}{py} \PY{o}{\PYZhy{}}\PY{o}{\PYZhy{}}\PY{n}{help}
          
          \PY{n}{cd} \PY{o}{/}\PY{n}{projectnb}\PY{o}{/}\PY{n}{landsat}\PY{o}{/}\PY{n}{users}\PY{o}{/}\PY{n}{ceholden}\PY{o}{/}\PY{l+m+mi}{2014}\PY{n}{\PYZus{}Landsat\PYZus{}Preprocess}\PY{o}{/}\PY{l+m+mi}{5}\PY{n}{\PYZus{}Stack}
          
          \PY{c}{\PYZsh{} Create our number of observations stack}
          \PY{n}{stack\PYZus{}nobs}\PY{o}{.}\PY{n}{py} \PY{o}{\PYZhy{}}\PY{o}{\PYZhy{}}\PY{n}{dname} \PY{l+s}{\PYZsq{}}\PY{l+s}{L*}\PY{l+s}{\PYZsq{}} \PY{o}{\PYZhy{}}\PY{o}{\PYZhy{}}\PY{n}{format} \PY{n}{GTiff} \PY{o}{.}\PY{o}{/} \PY{n}{stack\PYZus{}nobs}\PY{o}{.}\PY{n}{gtif} \PY{l+m+mi}{2} \PY{l+m+mi}{3} \PY{l+m+mi}{4}
          
          \PY{c}{\PYZsh{} Make it smaller!}
          \PY{n}{gdal\PYZus{}translate} \PY{o}{\PYZhy{}}\PY{n}{scale} \PY{l+m+mi}{0} \PY{l+m+mi}{4} \PY{l+m+mi}{0} \PY{l+m+mi}{255} \PY{o}{\PYZhy{}}\PY{n}{outsize} \PY{l+m+mi}{25}\PY{o}{\PYZpc{}} \PY{l+m+mi}{25}\PY{o}{\PYZpc{}} \PY{o}{\PYZhy{}}\PY{n}{of} \PY{n}{PNG} \PY{n}{stack\PYZus{}nobs}\PY{o}{.}\PY{n}{gtif} \PY{n}{stack\PYZus{}nobs}\PY{o}{.}\PY{n}{png}
          
          \PY{c}{\PYZsh{} Create class height file for gdaldem}
          \PY{n}{echo} \PY{l+m+mi}{0}\PY{p}{,} \PY{l+m+mi}{215}\PY{p}{,} \PY{l+m+mi}{25}\PY{p}{,} \PY{l+m+mi}{28} \PY{o}{\PYZgt{}} \PY{n}{colors}\PY{o}{.}\PY{n}{txt}
          \PY{n}{echo} \PY{l+m+mi}{1}\PY{p}{,} \PY{l+m+mi}{253}\PY{p}{,} \PY{l+m+mi}{174}\PY{p}{,} \PY{l+m+mi}{97} \PY{o}{\PYZgt{}\PYZgt{}} \PY{n}{colors}\PY{o}{.}\PY{n}{txt}
          \PY{n}{echo} \PY{l+m+mi}{2}\PY{p}{,} \PY{l+m+mi}{255}\PY{p}{,} \PY{l+m+mi}{255}\PY{p}{,} \PY{l+m+mi}{191} \PY{o}{\PYZgt{}\PYZgt{}} \PY{n}{colors}\PY{o}{.}\PY{n}{txt}
          \PY{n}{echo} \PY{l+m+mi}{3}\PY{p}{,} \PY{l+m+mi}{166}\PY{p}{,} \PY{l+m+mi}{217}\PY{p}{,} \PY{l+m+mi}{106} \PY{o}{\PYZgt{}\PYZgt{}} \PY{n}{colors}\PY{o}{.}\PY{n}{txt}
          \PY{n}{echo} \PY{l+m+mi}{4}\PY{p}{,} \PY{l+m+mi}{26}\PY{p}{,} \PY{l+m+mi}{150}\PY{p}{,} \PY{l+m+mi}{65} \PY{o}{\PYZgt{}\PYZgt{}} \PY{n}{colors}\PY{o}{.}\PY{n}{txt}
          
          \PY{n}{gdaldem} \PY{n}{color}\PY{o}{\PYZhy{}}\PY{n}{relief} \PY{n}{stack\PYZus{}nobs}\PY{o}{.}\PY{n}{gtif} \PY{n}{colors}\PY{o}{.}\PY{n}{txt} \PY{n}{stack\PYZus{}nobs\PYZus{}colormap}\PY{o}{.}\PY{n}{gtif}
          
          \PY{n}{gdal\PYZus{}translate} \PY{o}{\PYZhy{}}\PY{n}{outsize} \PY{l+m+mi}{25}\PY{o}{\PYZpc{}} \PY{l+m+mi}{25}\PY{o}{\PYZpc{}} \PY{o}{\PYZhy{}}\PY{n}{of} \PY{n}{PNG} \PY{n}{stack\PYZus{}nobs\PYZus{}colormap}\PY{o}{.}\PY{n}{gtif} \PY{n}{stack\PYZus{}nobs\PYZus{}colormap}\PY{o}{.}\PY{n}{png}
\end{Verbatim}

    \begin{Verbatim}[commandchars=\\\{\}]
Number of Observations in Stacks

Usage:
    stack\_nobs.py [options] <location> <output> [<maskvalues>\ldots]

Options:
    -n --name <name>        Pattern of each stack file [default: *stack]
    -d --dname <dname>      Pattern for each stack directory [default: LND*]
    -m --mask <band>        Mask band [default: 8]
    -n --ndv <ndv>          No data value [default: 255]
    -f --format <format>    Output file format [default: GTiff]
    -v --debug              Show (verbose) debugging messages
    -h --help               Show help
Finished image 1/4
Finished image 2/4
Finished image 3/4
Finished image 4/4
Input file size is 8121, 7101
0\ldots10\ldots20\ldots30\ldots40\ldots50\ldots60\ldots70\ldots80\ldots90\ldots100 - done.
0\ldots10\ldots20\ldots30\ldots40\ldots50\ldots60\ldots70\ldots80\ldots90\ldots100 - done.
Input file size is 8121, 7101
0\ldots10\ldots20\ldots30\ldots40\ldots50\ldots60\ldots70\ldots80\ldots90\ldots100 - done.
    \end{Verbatim}

     

    Visualize time series

\begin{quote}
module load CCDCTools/\_beta
\end{quote}

Example:

 


    % Add a bibliography block to the postdoc
    
    
    
    \end{document}
